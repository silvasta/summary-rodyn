
\section{Rigid-body Manipulators - Fixed Base}
\begin{sstTitleBox}{
    Equation of Motion
  }
  \begin{sstFullFrame}
    \color{white} \[
      M(q)\ddot{q} + b(q,\dot{q}) + g(q) = \tau + J_c^T F_c
    \]
  \end{sstFullFrame}
  \begin{sstOnlyFrame}
    \begin{align*}
      M(q)         &: \text{Mass/inertia matrix (symmetric, positive
      definite)} \\
      b(q,\dot{q}) &: \text{Coriolis/centrifugal vector} =
      C(q,\dot{q})\dot{q} \\
      g(q)         &: \text{Gravity vector} \\
      \tau         &: \text{Joint torques/forces (actuators)} \\
      J_c^T F_c    &: \text{External contact forces mapped to joint space}
    \end{align*}

  \end{sstOnlyFrame}
  \begin{sstOnlyFrame}
    \textbf{Properties}
    $\dot{M} - 2C$ is skew-symmetric\\
    (useful for stability proofs)
  \end{sstOnlyFrame}
\end{sstTitleBox}

\ssthl{Exam tip: Derive for 2-3 DOF arms; compute components numerically.}

\subsection{Principle of Virtual Work (D’Alembert’s Principle)}

\textbf{For Dynamic Equilibrium:}\\
Virtual work $\delta W = 0$ for all virtual displacements $\delta q$.
\[
  \sum_i (F_i - m_i \ddot{r}_i) \cdot \delta r_i + \sum_j (T_j - I_j
  \dot{\omega}_j - \omega_j \times I_j \omega_j) \cdot \delta \theta_j = 0
\]
Applies Newton's laws in directions of possible motion.

\ssthl{Exam: Use to derive EoM for constrained systems.}

\subsection{Single Rigid Body Dynamics}

\textbf{Translational}\quad $m \ddot{r} = F$ (Newton)

\textbf{Rotational}\quad $I \dot{\omega} + \omega \times I \omega = T$ (Euler)

\textbf{In moving frame} Velocities/accelerations include Coriolis terms

\ssthl{Exam: Compute for link in chain;\\
relate to kinematics (e.g., $^I v = ^I \dot{r} + \omega \times r$)}

\subsection{Newton-Euler Method}

Recursive computation for serial chains.
Efficient ($O(n)$)

\textbf{Forward}
Propagate velocities/accelerations, base to end-effectr

\textbf{Backward}
Propagate forces/torques from end-effector to base.

\textbf{Formulas} for link $i$:
\quad \quad
(Force/torque balance yields $\tau_i$)
\begin{align*}
  ^i v_i &= ^i R_{i-1}
  (^{i-1} v_{i-1} + ^{i-1} \omega_{i-1} \times ^{i-1} p_i)
  + ^i \dot{q}_i z_i\\
  ^i \omega_i &= ^i R_{i-1} ^{i-1} \omega_{i-1} + ^i \dot{q}_i z_i
  \quad (\text{revolute})
\end{align*}

\ssthl{exam: Apply to 3R arm, compare with Lagrangian}

\subsection{Projected Newton-Euler}

Principle of virtual work for multi-body systems

% TODO: L5.21 table

% WARN: EoM Definition L5.22

\subsection{Projected Newton-Euler}

Applies virtual work to multi-body systems.\\
Project dynamics into joint space.\\
Sum virtual works for EoM:
$\tau = \sum$ \text{projected inertias/forces}

\textbf{Key Terms}
Inertia projection, Coriolis, gravity via Jacobians.

\ssthl{Exam: Derive mass matrix $M_{ij} = \sum_k \text{trace}(J_k^T I_k J_k)$}

\subsection{Lagrange Formulation}

EoM from
$\frac{d}{dt} \frac{\partial L}{\partial \dot{q}}
- \frac{\partial L}{\partial q} = \tau$
\quad where $L = T - V$.

Energy:
\textbf{Kinetic} $T = \frac{1}{2} \dot{q}^T M(q) \dot{q}$
\ \
\textbf{Potential} $V = \sum m_i g^T r_i$
\begin{align*}
  M_{kl} &= \sum_{i=\max(k,l)}^n \text{trace} \left( \frac{\partial
  T_i}{\partial q_k} J_i \frac{\partial T_i^T}{\partial q_l} \right)
  + m_i \frac{\partial r_i^T}{\partial q_k} \frac{\partial r_i}{\partial q_l}
  \\
  b &= C\dot{q}\quad \text{from Christoffel symbols}
\end{align*}
\ssthl{Exam: Full derivation for planar 2R; identify $M, C, g$.}

\subsection{External Forces and Torques}

Map to joint torques:

\begin{minipage}[t]{0.5\textwidth}
  \textbf{Forces}
  \[
    \tau_{F_{ext}} = \sum_{j=1}^{n_f} J_{P,j}^T F_{ext,j}
  \]
\end{minipage}%
\begin{minipage}[t]{0.5\textwidth}
  \textbf{Torques}
  \[
    \tau_{T_{ext}} = \sum_{k=1}^{n_m} J_{R,k}^T T_{ext,k}
  \]
\end{minipage}
\textbf{Actuators}
\[
  \tau_{a} = \sum_k (J_{S_k} - J_{S_{k-1}})^T F_{a,k} + (J_{R_k} -
  J_{R_{k-1}})^T T_{a,k}
\]

$J_P, J_R$: Position/rotation Jacobians.

\ssthl{Exam: Compute for end-effector force.}

\subsection{Velocity in Moving Bodies}

Velocity in frame $i$:
\quad
\textbf{Linear}
$^i v = ^i \dot{r} + ^i \omega \times ^i r$
\quad
\textbf{Angular} $^i \omega$.

\textbf{Twist vector}: $V =
\begin{bmatrix} v \\ \omega
\end{bmatrix}$.

\textbf{Propagation}
$^i V_i = ^i A_{i-1} ^{i-1} V_{i-1} + ^i \dot{q}_i e_i$
\quad
($A$: adjoint).

\ssthl{Exam: Use in NE recursion}

\subsection{Jacobians for Prismatic/Revolute Joints}

Jacobian $J =
\begin{bmatrix} J_v & J_\omega
\end{bmatrix}$
maps $\dot{q} \to$ task velocity

\textbf{Prismatic} for joint $i$:
Col $i$ of $J_v = z_{i-1}$, $J_\omega = 0$

\textbf{Revolute}:
$J_v = z_{i-1} \times (p - p_{i-1})$, $J_\omega = z_{i-1}$

\ssthl{Exam example: 2R planar arm Jacobian,
singularity when det($J$)=0 (e.g., extended arm).}

\section{Dynamic Control}

\textbf{Control loops}\\
Position (inner vel/torque),
Torque (feedforward dynamics).

\ssthl{Exam: Block diagrams for PD + gravity comp:\\
$\tau = g(q) + K_p e + K_d \dot{e}$}

\subsection{Joint Impedance Control}

$\tau = g(q)
+ K_p (q_d - q)
+ K_d (\dot{q}_d - \dot{q})
+ K_i \int e \, dt + J^T F_{ext}$

Tuned as mass-spring-damper:\\
\textbf{Eigenfreq} $\omega = \sqrt{K_p/m}$
\textbf{Damping} $\zeta = K_d/(2\sqrt{m K_p})$

\ssthl{Exam: Stability via Lyapunov.}

\subsection{Inverse Dynamics Control (Computed Torque)}

$\tau = M(q) (\ddot{q}_d + K_d \dot{e} + K_p e) + b(q,\dot{q}) + g(q)$

Decouples to $\ddot{e} + K_d \dot{e} + K_p e = 0$

\ssthl{Exam: Derive error dynamics; choose gains for crit. damping.}

\subsection{Task-Space Dynamic Control}

\textbf{EoM in task space}
$\Lambda(x) \ddot{x} + \mu(x,\dot{x}) + p(x) = F + J^{-T} \tau_{ext}$

\textbf{Control}
$F = \Lambda (\ddot{x}_d + K_d \dot{e}_x + K_p e_x) + \mu + p$

\textbf{For redundancy} Use $J^\dagger = W^{-1} J^T (J W^{-1} J^T)^{-1}$
(weighted pseudo-inv)

\textbf{Multiple tasks}
Stack Jacobians, null-space projectors $N = I - J^\dagger J$

\ssthl{Exam: Prioritize tasks (e.g., motion > posture)}

\subsection{End-Effector Dynamics}

$\Lambda = (J M^{-1} J^T)^{-1}$
control as above.

Feedforward for traj: $\ddot{x}_d$ from planning.

\ssthl{Exam: Compute $\Lambda$ for 3R arm.}

\section{Interaction Control}

\subsection{Operational Space Control}

Unified: Includes force $F_c$ in dynamics.

% # TODO: write on 2 lines with some align* method as it doesn't fit on 1 line
$\tau = J^T \Lambda (\ddot{x}_d + K_d \dot{e}_x + K_p e_x -
J M^{-1} (b + g)) + (I - J^T \bar{J}^T) \tau_0$

\ssthl{Exam: Hybrid force-motion via selection matrix $S$
(e.g., $S=I$ motion, $S=0$ force)}

\subsection{Selection Matrix}

$S$: Diagonal, separates DOFs (e.g., force in z, motion in x-y)
Control: Blend impedances.

\ssthl{Exam: Formula for hybrid:
$\tau = J^T (S F_m + (I-S) F_f)$.}

% HACK: Operational Space Control - FINAL big picture? P.30

\subsection{Inverse Dynamics as QP}

\textbf{Formulation}
$\min_u \| A u - b \|^2_W$ s.t. constraints (torque limits,...)

\textbf{For tasks}
Hierarchical QPs, solve sequentially

\ssthl{Exam: Least-squares for overconstrained systems,\\
  weighted pseudo-inv
$J^\dagger_W = W^{-1} J^T (J W^{-1} J^T)^{-1}$}

\section{Floating Base Dynamics}

For mobile/legged robots: Unactuated base.

\subsection{Generalized coordinates}
\[
  q =
  \begin{pmatrix}
    q_b\\q_j
  \end{pmatrix}
  \quad\text{with}\quad
  q_b =
  \begin{pmatrix}
    q_{b_P}\\q_{b_R}
  \end{pmatrix}
  \ \in\
  \mathbb{R}^{3}\times SO(3)
\]

\subsection{Generalized velocities and accelerations}
\[
  u =
  \begin{pmatrix}
    _Iv_B\\
    _B\omega_IB\\
    \dot{\varphi}_1\\
    \vdots\\
    \dot{\varphi}_{n_j}\\
  \end{pmatrix}
  \quad \dot{u} =
  \begin{pmatrix}
    _Ia_B\\
    _B\psi_IB\\
    \ddot{\varphi}_1\\
    \vdots\\
    \ddot{\varphi}_{n_j}\\
  \end{pmatrix}
  \ \in\
  \mathbb{R}^{6+n_j}=
  \mathbb{R}^{n_u}
\]
\[
  u= E_{fb}\cdot\dot{q}
  \quad\text{with}\quad
  E_{fb}=
  \begin{bmatrix}
    1_{3\times3}&0&0\\
    0&E_{\chi_R}&0\\
    0&0&1_{n_j\times n_j}
  \end{bmatrix}
\]
$E_{fb}$ maps quaternions/Euler to twists.

\ssthl{Exam: Note $\dot{q} \neq u$ due to SO(3).}

\subsection{Differential Kinematics}

Floating:
$J =
\begin{bmatrix} J_b & J_j
\end{bmatrix}$,\
$\dot{x} = J(q) u$
(task vel from gen. vel)

\subsection{Contacts and Constraints}

\textbf{Hard} $J_c u = 0$ (no slip),\
\textbf{Soft} Compliant models

\textbf{Constraints}
$J_c \dot{u} + \dot{J}_c u = 0$

\ssthl{Exam: Enforce via Lagrange multipliers $F_c = -\lambda$.}

\section{Dynamics of Floating Base Systems}

% HACK: extend EoM with new stuff

\begin{sstFullFrame}
  \color{white}
  \[
    M(q) \dot{u} + h(q,u) = S^T \tau + J_c^T F_c
  \]
\end{sstFullFrame}

where $S =
\begin{bmatrix} 0 & I
\end{bmatrix}$ (underactuated base), $h = b + g$.

\subsection{Constraint-Consistent Dynamics}

Project to null-space of constraints:
$\bar{M} \dot{\bar{u}} + \bar{h} = \bar{S}^T \tau$.

\ssthl{Exam: For legged, balance via CoM control}

\subsection{Contact Dynamics}

\textbf{Impacts}
Instant velocitiy change:

$\Delta u = - (J_c M^{-1} J_c^T)^{-1} J_c u^- $
(pre-impact)

\textbf{Soft}
Spring-damper
$F_c = k \delta + d \dot{\delta}$

\subsection{Dynamic Control Methods}

\textbf{Multi-task}
Motion as tasks (e.g., CoM, feet)

\textbf{Internal forces}
Null-space torques for stability.

\subsection{Control Using Inverse Dynamics}

$\tau = S^+ (M \dot{u}_d + h - J_c^T F_{c,d}) + N \tau_0$

\ssthl{Exam: Choose $\dot{u}_d$ for tasks}

\subsection{Task Space Control as Quadratic Program}

Hierarchical QP: Min cost for primary task, then secondary in
null-space.

\ssthl{Exam: Formulate for torque optimization with friction
cone constraints}

