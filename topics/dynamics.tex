
\begin{sstTitleBox}{
    Equations of Motion
  }
  \begin{sstOnlyFrame}
    \[
      M(q)\ddot{q} +
      b(q,\dot{q}) +
      g(q) =
      \tau +
      J_c^T F_c
    \]
  \end{sstOnlyFrame}

  \begin{sstOnlyFrame}
    \begin{align*}
      & M(q)         &  & \text{Mass matrix}                     \\
      & \ddot{q}     &  & \text{Generalized coordinates}         \\
      & b(q,\dot{q}) &  & \text{Centrifugal and Coriolis forces} \\
      & g(q)         &  & \text{Gravity forces}                  \\
      & \tau         &  & \text{Generalized forces}              \\
      & F_c          &  & \text{External forces}                 \\
      & J_c          &  & \text{Contact Jacobian}                \\
    \end{align*}
  \end{sstOnlyFrame}
\end{sstTitleBox}

\section{Principle of Virtual Work}
- Principle of virtual work (D’Alembert’s Principle)

- Dynamic equilibrium imposes zero virtual work (for all virtual displacements)

- Newton’s law for every particle in direction it can move

\section{Single  Rigid Body}

% TODO: check Single Rigid Body,also with kinematics

\section{Newton-Euler Method}

% AI: short description

\section{Projected Newton-Euler}

Principle of virtual work for multi-body systems

% TODO: L5.21 table

% WARN: EoM Definition L5.22

\section{Lagrange II}

- Kinetic energy

- Potential energy

% TODO: check formula L5.26

\section{External Forces and Torques}

% TODO: layout

\begin{align*}
  \textbf{Forces}     \tau_{F_{ext}} & = \sum_{j=1}^{n_{f,ext}}
  J_{P,j}^T T_{ext,k}
  \\
  \textbf{Torques}    \tau_{T_{ext}} & = \sum_{k=1}^{n_{m,ext}}
  J_{R,k}^T T_{ext,k}
  \\
  \textbf{Actuators}  \tau_{a,k}     & = (J_{S_k} - J_{S_{k-1}})^T
  F_{a,k} + (J_{R_k} - J_{R_{k-1}})^T T_{a,k}
\end{align*}

\section{Velocity in Moving Bodies}

- Definitions

- Moving Frame

\section{Prismatic Joints}

TODO: Jacobians

- Position Jacobian

- Rotation Jacobian

- Example

\section{Dynamic Control}

% TODO: images position/torque control L6.6

\subsubsection{Joint Impedance Control}

% TODO: Formula

% HACK: Formula with Gravity compensation

\subsubsection{Inverse Dynamics Control}

Compensate for system dynamics + PD law on acceleration

% TODO: copy different formulas

- every joint behaves like decoupled mass-spring damper

- Eigenfrequency

- Damping

\section{Task-space Dynamics control}

- single task: just use pseudo-inverse

% WARN: formula Task-space dynamics control

- multiple task: stack $J_i,w_i$, pseudo-inverse, done (equal priority)

\subsection{End-effector dynamics}

% TODO: formula

- end-effector motion control

- trajectory control (feedforwad term)

\section{Interaction Control}

\subsection{Operational Space Control}

Generalized framework to control motion and force

% TODO: formula
- $F_c$ as contact force

\subsection{Selection Matrix}

- seperate motion and force directions

% TODO: check selection matrix

% HACK: Operational Space Control - FINAL big picture? P.30

\subsection{Inverse Dynamics as QP}

- use: some sort of “mass-matrix weighted pseudo-inverse”

- quadratic optimization

% TODO: check Least Square Optimization

- Solving set of QPs

\section{Floating Base Dynamics}

\subsection{Generalized coordinates}

\[
  q =
  \begin{pmatrix}
    q_b\\q_j
  \end{pmatrix}
  \quad\text{with}\quad
  q_b =
  \begin{pmatrix}
    q_{b_P}\\q_{b_R}
  \end{pmatrix}
  \ \in\
  \mathbb{R}^{3}\times SO(3)
\]

\subsection{Generalized velocities and accelerations}

\[
  u =
  \begin{pmatrix}
    _Iv_B\\
    _B\omega_IB\\
    \dot{\varphi}_1\\
    \vdots\\
    \dot{\varphi}_{n_j}\\
  \end{pmatrix}
  \quad \dot{u} =
  \begin{pmatrix}
    _Ia_B\\
    _B\psi_IB\\
    \ddot{\varphi}_1\\
    \vdots\\
    \ddot{\varphi}_{n_j}\\
  \end{pmatrix}
  \ \in\
  \mathbb{R}^{6+n_j}=
  \mathbb{R}^{n_u}
\]

Very often, people write $\dot{q}$ but they mean $u$

\[
  u= E_{fb}\cdot\dot{q}
  \quad\text{with}\quad
  E_{fb}\cdot\dot{q}=
  \begin{bmatrix}
    1_{3\times3}&0&0\\
    0&E_{\chi_R}&0\\
    0&0&1_{n_j\times n_j}
  \end{bmatrix}
\]

\subsection{Differential kinematics}

% WARN: formula DK floating base

\section{Contacts and Constraints}

% TODO: constraints

\subsection{Properties of Contact Jacobian}

\section{Dynamics of Floating Base Systems}

% HACK: extend EoM with new stuff

- External Forces

- Soft Contact

- Hard Contact

\section{Constraint consistent dynamics}

% TODO: something usefull in formulas?

\subsection{Contact Dynamics}

% HACK: take impact formulas

\section{Dynamic Control Methods}

- Behavior as Multiple Tasks

- Internal Forces

\subsection{Control using inverse dynamics}

% TODO: formulas

\section{Task Space Control as Quadratic Program}

% NOTE: again  least  squares

