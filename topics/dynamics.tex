\section{Rigid-body Manipulators - Fixed Base}
\begin{sstTitleBox}{
    Equation of Motion
  }
  \begin{sstFullFrame}
    \color{white} \[
      M(q)\ddot{q} + b(q,\dot{q}) + g(q) = \tau + J_c^T F_c
    \]
  \end{sstFullFrame}
  \begin{sstOnlyFrame}
    \begin{align*}
      M(q)         &: \text{Mass/inertia matrix (symmetric, positive
      definite)} \\
      b(q,\dot{q}) &: \text{Coriolis/centrifugal vector} =
      C(q,\dot{q})\dot{q} \\
      g(q)         &: \text{Gravity vector} \\
      \tau         &: \text{Joint torques/forces (actuators)} \\
      J_c^T F_c    &: \text{External contact forces mapped to joint space}
    \end{align*}
  \end{sstOnlyFrame}
  \begin{sstOnlyFrame}
    \textbf{Properties}:
    $\dot{M} - 2C$ skew-symmetric\\
    Passivity: $\dot{q}^T (\dot{M}-2C)\dot{q}=0$,
    $M$ bounded/invertible,
    linear in parameters (for identification)
  \end{sstOnlyFrame}
\end{sstTitleBox}

\ssthl{Exam tip: Derive for 2-3 DOF arms; compute components numerically.}

\ssthl{Exam tip: Derive EoM for 2-3 DOF\\
  (e.g., 2R planar: compute M as 2x2, C via Christoffel, g from potentials)
numerical torque calc at given $q,\dot{q}$}

\subsection{Principle of Virtual Work (D’Alembert’s Principle)}

For dynamic equilibrium: Virtual work $\delta W = 0$ for all $\delta q$.
\[
  \sum_i (F_i - m_i \ddot{r}_i) \cdot \delta r_i + \sum_j (T_j - I_j
  \dot{\omega}_j - \omega_j \times I_j \omega_j) \cdot \delta \theta_j = 0
\]
Applies Newton's laws in directions of possible motion,\\
extends to constraints via multipliers.

\ssthl{Exam: Derive EoM for constrained systems (e.g., closed-chain);
relate to projected dynamics.}

\subsection{Single Rigid Body Dynamics}

\textbf{Translational}\quad $m \ddot{r} = F$ (Newton)

\textbf{Rotational}\quad $I \dot{\omega} + \omega \times I \omega = T$ (Euler)

Moving frame:
Include Coriolis/centrifugal vel.
$^I v = ^I \dot{r} + \omega \times r$

\ssthl{Exam: Compute for link in chain, propagate to next\\
(e.g., NE forward pass)}

\subsection{Newton-Euler Method}

Recursive for serial chains ($O(n)$ efficiency)\\

\textbf{Forward}: Velocities/accelerations base$\to$EE

\textbf{Backward}: Forces/torques EE$\to$base, yields $\tau_i$.

For link $i$ (revolute):
\begin{align*}
  ^i \omega_i &= ^i R_{i-1} ^{i-1} \omega_{i-1} + \dot{q}_i ^i z_i, \\
  ^i v_i &= ^i R_{i-1} (^{i-1} v_{i-1} + ^{i-1} \omega_{i-1} \times
  ^{i-1} p_i) + \dot{q}_i ^i z_i, \\
  ^i \dot{v}_i &= ^i R_{i-1} ^{i-1} \dot{v}_{i-1} + ^i \dot{\omega}_i
  \times ^i p_{c,i} + \dots (\text{full acc inc. Coriolis})
\end{align*}
Force:
$f_i = m_i \dot{v}_{c,i} + \omega_i \times ( \omega_i \times m_i r_{c,i} )$

\ssthl{Exam: Apply to 3R arm; compare Lagrangian (same EoM, NE faster
for high DOF).}

\subsection{Projected Newton-Euler}

Applies virtual work to multi-body systems.\\
Project dynamics into joint space.\\
EoM: $\tau = \sum$ projected inertias/forces (via Jacobians).\\
Mass: $M_{ij} = \sum_k \text{trace}(J_{v,k}^T m_k J_{v,k} +
J_{\omega,k}^T I_k J_{\omega,k})$.\\
Coriolis/gravity similarly projected.

\ssthl{Exam: Derive M for 2DOF; use trace identity for efficiency.}

\subsection{Lagrange Formulation}

EoM from
$\frac{d}{dt} \frac{\partial L}{\partial \dot{q}}
- \frac{\partial L}{\partial q} = \tau$
\quad where $L = T - V$.

Energy -
\textbf{Kinetic} $T = \frac{1}{2} \dot{q}^T M(q) \dot{q}$
\ \
\textbf{Potential} $V = \sum m_i g^T r_i$

\begin{align*}
  M_{kl} &= \sum_{i=\max(k,l)}^n \text{trace} \left( \frac{\partial
  T_i}{\partial q_k} J_i \frac{\partial T_i^T}{\partial q_l} \right)
  + m_i \frac{\partial r_i^T}{\partial q_k} \frac{\partial r_i}{\partial q_l}
  \\
  C_{kj} &= \sum_i \Gamma_{kji} \dot{q}_i
  \quad\text{(Coriolis)}
  \\
  \Gamma_{kji} &= \frac{1}{2} (\partial_k M_{ji} + \partial_j M_{ki}
  - \partial_i M_{kj})
  \quad\text{(Christoffel)}
\end{align*}

\ssthl{Exam: Full derivation for planar 2R, identify $M, C, g$\\
  M symmetric, C skew contrib,
ID params via regression $Y(q,\dot{q},\ddot{q}) \theta = \tau$.}

\ssthl{Exam: For 2R arm, derive M,C,g via Lagrange (energy) vs NE
(recursion); verify same $\tau$ at sample $q=(0,\pi/2), \dot{q}=(1,1)$}

\subsection{External Forces and Torques}

Map to joints: $\tau_{ext} = J_P^T F_{ext} + J_R^T T_{ext}$

\begin{minipage}[t]{0.5\columnwidth}
  \textbf{Forces}
  \[
    \tau_{F_{ext}} = \sum_{j=1}^{n_f} J_{P,j}^T F_{ext,j}
  \]
\end{minipage}%
\begin{minipage}[t]{0.5\columnwidth}
  \textbf{Torques}
  \[
    \tau_{T_{ext}} = \sum_{k=1}^{n_m} J_{R,k}^T T_{ext,k}
  \]
\end{minipage}
\textbf{Actuators}
\[
  \tau_{a} = \sum_k (J_{S_k} - J_{S_{k-1}})^T F_{a,k} + (J_{R_k} -
  J_{R_{k-1}})^T T_{a,k}
\]
$J_P, J_R$: Position/rotation Jacobians.

\ssthl{Exam: Compute for EE force, add to EoM.}

\subsection{Velocity in Moving Bodies}

\textbf{Linear}
$^i v = ^i \dot{r} + ^i \omega \times ^i r$
\quad
\textbf{Angular} $^i \omega$
\quad
(Velocity in frame $i$)

\textbf{Twist vector}: $V =
\begin{bmatrix} v \\ \omega
\end{bmatrix}$

\textbf{Propagation}
$^i V_i = ^i A_{i-1} ^{i-1} V_{i-1} + ^i \dot{q}_i e_i$
\quad
($A$: adjoint).

\ssthl{Exam: Use in NE, relate to Jacobian columns}

\subsection{Jacobians for Prismatic/Revolute Joints}

Jacobian
$J =
\begin{bmatrix} J_v & J_\omega
\end{bmatrix}$
maps $\dot{q} \to$ task velocity,
$\dot{x} = J \dot{q}$

\textbf{Singularity} $\det(J J^T)=0$

\textbf{Prismatic} $J_{v,i} = z_{i-1}$, $J_{\omega,i} = 0$

\textbf{Revolute}$i$:
$J_{v,i} = z_{i-1} \times (p - p_{i-1})$, $J_{\omega,i} = z_{i-1}$.\\

\ssthl{Exam: 2R planar Jacobian, singularity when aligned/extended}

\ssthl{Exam: Compute J for RRP (SCARA); singularities at det(J)=0
(e.g., arm folded); manipulability $\sqrt{\det(J J^T)}$.}

\section{Dynamic Control}

\textbf{Control loops}\\
Position (inner velocity/torque),
Torque (feedforward dynamics).

\ssthl{Exam: Block diagrams for PD + gravity comp:\\
  $\tau = g(q) + K_p e + K_d \dot{e}$,
error $\ddot{e} + K_d \dot{e} + K_p e = M^{-1} \delta \tau$.}

\subsection{Joint Impedance Control}

$\tau = g(q)
+ K_p (q_d - q)
+ K_d (\dot{q}_d - \dot{q})
+ K_i \int e \, dt + J^T F_{ext}$

As mass-spring-damper:
$\omega_n
= \sqrt{K_p / m}$, $\zeta
= K_d / (2 \sqrt{m K_p})$

\ssthl{Exam: Lyapunov stability
  $V = \frac{1}{2} \dot{e}^T M \dot{e}
+ \frac{1}{2} e^T K_p e \to \dot{V} \leq 0$}

\ssthl{Exam: Prove stability for PD control using Lyapunov\\
  $V=\frac{1}{2}\dot{q}^T M \dot{q} + \frac{1}{2} e^T K_p e$,
  $\dot{V} = -\dot{q}^T K_d \dot{q} \leq 0$\\
(LaSalle for convergence)}

\subsection{Inverse Dynamics Control (Computed Torque)}

$\tau = M(q) (\ddot{q}_d + K_d \dot{e} + K_p e) + b(q,\dot{q}) + g(q)$

Decouples: $\ddot{e} + K_d \dot{e} + K_p e = 0$,
crit damp $K_d = 2 \sqrt{K_p}$

\ssthl{Exam: Derive error dynamics;
choose gains for crit. damping, f.e. overshoot <5\% }

\subsection{Task-Space Dynamic Control}

\textbf{EoM}
\[
  \Lambda(x) \ddot{x} + \mu(x,\dot{x}) + p(x) = F + J^{-T} \tau_{ext}
\]
with $\Lambda = (J M^{-1} J^T)^{-1}$

\textbf{Control}
$F = \Lambda (\ddot{x}_d + K_d \dot{e}_x + K_p e_x) + \mu + p$

\textbf{Redundancy} weighted psd-inv:
$J^\dagger = W^{-1} J^T (J W^{-1} J^T)^{-1}$

\textbf{Null-space projector} $N = I - J^\dagger J$

\textbf{Multiple tasks}
Stack Jacobians, project secondary to $N$

\ssthl{Exam: Prioritize (eg. EE motion > joint limits)
compute $\Lambda$ for 3R}

\subsection{End-Effector Dynamics}

As above and with feedforward $\ddot{x}_d$ from trajectory planning.

\ssthl{Exam: Hybrid with selection S (diag, 0=force,1=motion)}

\section{Interaction Control}

\subsection{Operational Space Control}
\begin{align*}
  \tau =
  &J^T \Lambda (\ddot{x}_d + K_d \dot{e}_x + K_p e_x - J M^{-1} (b + g))
  \\
  &+ (I - J^T \bar{J}^T) \tau_0\\
  \bar{J} =& \Lambda^{-1} J M^{-1}
\end{align*}

\ssthl{Exam: Formula for hybrid:
$\tau = J^T (S F_m + (I-S) F_f)$}

\subsection{Selection Matrix}

$S$: Diagonal, separates DOFs (e.g., force in z, motion in x-y)
Control: Blend impedances.

% HACK: Operational Space Control - FINAL big picture? P.30

\ssthl{Exam: Hybrid f-m via $S$ (e.g., $S=I$ motion, $S=0$ force)}

\subsection{Inverse Dynamics as QP}

\textbf{Formulation}
$\min_u \| A u - b \|^2_W$ s.t. constraints (torque limits,...)

\textbf{Hierarchical}
Solve primary, project secondary to null

\ssthl{Exam: Formulate for redundancy; weighted pseudo-inv for LS}

\ssthl{Exam: Formulate QP for 7DOF arm:
  min $\| \dot{q} \|$ s.t. $J \dot{q} = \dot{x}_d$,
torque bounds; null for secondary (e.g., obstacle avoid).}

\section{Floating Base Dynamics}

For mobile/legged robots: Unactuated base.

\subsection{Generalized coordinates}
\[
  q =
  \begin{pmatrix}
    q_b\\q_j
  \end{pmatrix}
  \quad\text{with}\quad
  q_b =
  \begin{pmatrix}
    q_{b_P}\\q_{b_R}
  \end{pmatrix}
  \ \in\
  \mathbb{R}^{3}\times SO(3)
\]

\subsection{Generalized Velocities/Accelerations}

Twist-based: $u = [^I v_b^T \, ^b \omega_b^T \, \dot{q}_j^T]^T \in
\mathbb{R}^{6+n_j}$; $\dot{u}$ similar. Map: $u = E_{fb} \dot{q}$
($E_{fb}$ handles rot param, e.g., quats to ang vel).

\ssthl{Exam: Note $\dot{q} \neq u$ due to SO(3); use for non-holonomic systems.}

\subsection{Generalized velocities and accelerations}
\[
  u =
  \begin{pmatrix}
    _Iv_B\\
    _B\omega_IB\\
    \dot{\varphi}_1\\
    \vdots\\
    \dot{\varphi}_{n_j}\\
  \end{pmatrix}
  \quad \dot{u} =
  \begin{pmatrix}
    _Ia_B\\
    _B\psi_IB\\
    \ddot{\varphi}_1\\
    \vdots\\
    \ddot{\varphi}_{n_j}\\
  \end{pmatrix}
  \ \in\
  \mathbb{R}^{6+n_j}=
  \mathbb{R}^{n_u}
\]
\[
  u= E_{fb}\cdot\dot{q}
  \quad\text{with}\quad
  E_{fb}=
  \begin{bmatrix}
    1_{3\times3}&0&0\\
    0&E_{\chi_R}&0\\
    0&0&1_{n_j\times n_j}
  \end{bmatrix}
\]
$E_{fb}$ maps quaternions/Euler to twists.

\ssthl{Exam: Note $\dot{q} \neq u$ due to SO(3).}

\subsection{Differential Kinematics}

Floating:
$J = [J_b \ J_j]$, $\dot{x} = J(q) u$
(task vel from gen. vel)

\subsection{Contacts and Constraints}

\textbf{Hard}
$J_c u = 0$ (no-slip)
\quad
\textbf{Const. acc.}
$J_c \dot{u} + \dot{J}_c u = 0$

\textbf{Soft}
$F_c = k \delta + d \dot{\delta}$
\quad
\textbf{Friction}
Cone $|F_t| \leq \mu F_n$

\ssthl{Exam: Enforce via multipliers $\lambda = -F_c$ \\
impacts $\Delta u = -(J_c M^{-1} J_c^T)^{-1} J_c u^-$}

\section{Dynamics of Floating Base Systems}

\begin{sstFullFrame}
  \color{white}
  \[
    \textbf{EoM}\quad
    M(q) \dot{u} + h(q,u) = S^T \tau + J_c^T F_c
  \]
\end{sstFullFrame}

where $S =
\begin{bmatrix} 0 & I
\end{bmatrix}$ (underactuated base),
$h = C u + g$.

Centroidal: $A_G \dot{u} + \dot{A}_G u = \sum F_{ext} + g_G$

(CMM $A_G$ for momentum).

\ssthl{Exam: Project to constraint null; CoM control for balance}

\ssthl{Exam: Derive centroidal momentum for quadruped; control CoM
vel via $A_G u$ for balance under disturbances.}

\subsection{Constraint-Consistent Dynamics}

Project to null-space of constraints:
$\bar{M} \dot{\bar{u}} + \bar{h} = \bar{S}^T \tau$.

\ssthl{Exam: For legged, balance via CoM control}

\subsection{Contact Dynamics}

\textbf{Impacts}
Instant velocitiy change:

$\Delta u = - (J_c M^{-1} J_c^T)^{-1} J_c u^- $
(pre-impact)

% TODO: check

\textbf{Soft}
Spring-damper
$F_c = k \delta + d \dot{\delta}$

\subsection{Dynamic Control Methods}

\textbf{Multi-task}
CoM, feet as priorities.

Inv dyn: $\tau = S^+ (M \dot{u}_d + h - J_c^T F_{c,d}) + N \tau_0$

\ssthl{Exam: $\dot{u}_d$ from tasks; QP for torque opt w/ cones.}

