% TODO: import .md

\section{Lecture 6 - Introduction Dynamic Control}

\subsection{Position vs. Torque Controlled Robot Arms}

% TODO: images position/torque control

\subsection{Joint Impedance and Inverse Dynamics Control}

\subsubsection{Joint Impedance Control}

- Formula

- Formula with Gravity compensation

\subsubsection{Inverse Dynamics Control}

\textbf{Compensate for system dynamics + PD law on acceleration}

% TODO: copy different formulas

- every joint behaves like decoupled mass-spring damper

- Eigenfrequency

- Damping

\subsubsection{Task-space dynamics control}

- single task: just use pseudo-inverse

- multiple task: stack $J_i,w_i$, pseudo-inverse, done (equal priority)

\subsection{Task-space Dynamics Control}

% TODO: look at "big picture so far"

- end-effector dynamics

- end-effector motion control

- trajectory control (feedforwad term)

\subsection{Interaction Control}

\subsubsection{Operational Space Control}

- $F_c$ as contact force

\subsubsection{Selection Matrix}

- seperate motion and force directions

% TODO: check selection matrix

% HACK: Operational Space Control - FINAL big picture? P.26

\subsection{Inverse Dynamics as QP}

- use: some sort of “mass-matrix weighted pseudo-inverse”

- quadratic optimization

% TODO: check Least Square Optimization

- Solving set of QPs

