\section{Rigid-body Manipulators - Fixed Base}
\begin{sstTitleBox}{
    Equation of Motion
  }
  \begin{sstFullFrame}
    \color{white} \[
      M(q)\ddot{q} + b(q,\dot{q}) + g(q) = \tau + J_c^T F_c
    \]
  \end{sstFullFrame}
  \begin{sstOnlyFrame}
    \begin{align*}
      M(q)         &: \text{Mass/inertia matrix (symmetric, positive
      definite)} \\
      b(q,\dot{q}) &: \text{Coriolis/centrifugal vector} =
      C(q,\dot{q})\dot{q} \\
      g(q)         &: \text{Gravity vector} \\
      \tau         &: \text{Joint torques/forces (actuators)} \\
      J_c^T F_c    &: \text{External contact forces mapped to joint space}
    \end{align*}
  \end{sstOnlyFrame}
  \begin{sstOnlyFrame}
    \textbf{Properties}:
    $\dot{M} - 2C$ skew-symmetric\\
    Passivity: $\dot{q}^T (\dot{M}-2C)\dot{q}=0$,
    $M$ bounded/invertible,
    linear in parameters (for identification)
  \end{sstOnlyFrame}
\end{sstTitleBox}

\subsection{Principle of Virtual Work (D’Alembert’s Principle)}

For dynamic equilibrium: Virtual work $\delta W = 0$ for all $\delta q$.
\[
  \sum_i (F_i - m_i \ddot{r}_i) \cdot \delta r_i + \sum_j (T_j - I_j
  \dot{\omega}_j - \omega_j \times I_j \omega_j) \cdot \delta \theta_j = 0
\]
Applies Newton's laws in directions of possible motion,\\
extends to constraints via multipliers.

\ssthl{Exam: Derive EoM for constrained systems (e.g., closed-chain);
relate to projected dynamics.}

\subsection{Single Rigid Body Dynamics}

\textbf{Translational}\quad $m \ddot{r} = F$ (Newton)

\textbf{Rotational}\quad $I \dot{\omega} + \omega \times I \omega = T$ (Euler)

Moving frame:
Include Coriolis/centrifugal vel.
$^I v = ^I \dot{r} + \omega \times r$

\subsection{Newton-Euler Method}

Recursive for serial chains ($O(n)$ efficiency)\\

\textbf{Forward}: Velocities/accelerations base$\to$EE

\textbf{Backward}: Forces/torques EE$\to$base, yields $\tau_i$.

For link $i$ (revolute):
\begin{align*}
  ^i \omega_i &= ^i R_{i-1} ^{i-1} \omega_{i-1} + \dot{q}_i ^i z_i, \\
  ^i v_i &= ^i R_{i-1} (^{i-1} v_{i-1} + ^{i-1} \omega_{i-1} \times
  ^{i-1} p_i) + \dot{q}_i ^i z_i, \\
  ^i \dot{v}_i &= ^i R_{i-1} ^{i-1} \dot{v}_{i-1} + ^i \dot{\omega}_i
  \times ^i p_{c,i} + \dots (\text{full acc inc. Coriolis})
\end{align*}
Force:
$f_i = m_i \dot{v}_{c,i} + \omega_i \times ( \omega_i \times m_i r_{c,i} )$

\subsection{Projected Newton-Euler}

The Projected Newton-Euler method uses the Principle of Virtual Work
to project the Cartesian Newton-Euler equations of individual links
into the space of generalized coordinates $\mathbf{q}$.\\
This automatically eliminates internal constraint forces.

\textbf{Mass Matrix Projection}
\[ \mathbf{M}(\mathbf{q}) = \sum_{k=1}^{n_L} \left(
    \mathbf{J}_{v,k}^T m_k \mathbf{J}_{v,k} + \mathbf{J}_{\omega,k}^T
\mathbf{I}_k \mathbf{J}_{\omega,k} \right) \]

\textbf{Nonlinear Terms (Coriolis/Centrifugal/Gravity)}\\
Let $\mathbf{h} = \mathbf{C}\dot{\mathbf{q}} + \mathbf{g}$.
This vector is the sum of link wrenches mapped to joint space
via the transpose of the Jacobians:
\[ \mathbf{h} = \sum_{k=1}^{n_L} \left( \mathbf{J}_{v,k}^T
\mathbf{F}_k + \mathbf{J}_{\omega,k}^T \mathbf{T}_k \right) \]
Link-wise Newton-Euler forces/torques (evaluated at $\ddot{\mathbf{q}}=0$):
\begin{align*}
  \mathbf{F}_k &= m_k \dot{\mathbf{v}}_\text{k, rem} - m_k \mathbf{g}_0 \\
  \mathbf{T}_k &= \mathbf{I}_k \dot{\boldsymbol{\omega}}_\text{k, rem} +
  \boldsymbol{\omega}_k \times \mathbf{I}_k \boldsymbol{\omega}_k
\end{align*}
$\dot{\mathbf{v}}_\text{rem}$ and
$\dot{\boldsymbol{\omega}}_\text{rem}$ refer to convective
acceleration terms (i.e., $\dot{\mathbf{J}}\dot{\mathbf{q}}$)

\textbf{Virtual Work Derivation:}
The projection is valid because $\delta \mathbf{q}^T (
\mathbf{M}\ddot{\mathbf{q}} + \mathbf{h} - \boldsymbol{\tau} ) = 0$
for all virtual displacements $\delta \mathbf{q}$ that comply with
the constraints.

\subsection{Projected Newton-Euler}

Combines Newton-Euler with Lagrange: Uses virtual work in generalized
coordinates to project Cartesian dynamics into joint space,
automatically handling constraints. Most practical for robotics as it
yields EoM in standard form efficiently.

EoM: $\tau = \sum$
projected inertias/forces (via Jacobians)\\
Mass: $M_{ij} = \sum_k \text{trace}(J_{v,k}^T m_k J_{v,k} +
J_{\omega,k}^T I_k J_{\omega,k})$.\\
Coriolis/gravity similarly projected (e.g., $b_i = \sum_k J_{v,k}^T
  (m_k \dot{v}_k + \omega_k \times m_k v_k) + J_{\omega,k}^T (\Theta_k
\dot{\omega}_k + \omega_k \times \Theta_k \omega_k)$).

Derivation from virtual work: $\delta q^T (M \ddot{q} + b + g - \tau)
= 0$ for constraint-compliant $\delta q$.

\subsection{Lagrange Formulation}

EoM from Euler-Lagrange equation:
$\frac{d}{dt} \frac{\partial L}{\partial \dot{q}}
- \frac{\partial L}{\partial q} = \tau$
\quad where $L = T - V$ (Lagrangian).

\textbf{Kinetic energy} $T = \frac{1}{2} \dot{q}^T M(q) \dot{q} =
\sum_i \frac{1}{2} m_i v_i^2 + \frac{1}{2} \omega_i^T I_i \omega_i$.\\
\textbf{Potential energy} $V = \sum_i m_i g^T r_i$ (or $U = \sum m_i g h_i$).

Mass matrix:
\begin{align*}
  M_{kl} &= \sum_{i=\max(k,l)}^n \text{trace} \left( \frac{\partial
  T_i}{\partial q_k} J_i \frac{\partial T_i^T}{\partial q_l} \right)
  + m_i \frac{\partial r_i^T}{\partial q_k} \frac{\partial r_i}{\partial q_l}
\end{align*}

Coriolis matrix via Christoffel symbols:
\begin{align*}
  C_{kj} &= \sum_i \Gamma_{kji} \dot{q}_i, \\
  \Gamma_{kji} &= \frac{1}{2} (\partial_k M_{ji} + \partial_j M_{ki}
  - \partial_i M_{kj})
\end{align*}

Gravity: $g_k = -\frac{\partial V}{\partial q_k}$.

Computation steps:
\begin{itemize}
  \item Forward kinematics for positions $r_i$,
    velocities $v_i$, $\omega_i$
  \item Form $T$ and $V$ in terms of $q, \dot{q}$
  \item Apply Euler-Lagrange to get $M, C, g$
\end{itemize}

\subsection{External Forces and Torques}

Map to joints: $\tau_{ext} = J_P^T F_{ext} + J_R^T T_{ext}$

$J_P, J_R$: Position/rotation Jacobians.

\begin{minipage}[t]{0.5\columnwidth}
  \textbf{Forces}
  \[
    \tau_{F_{ext}} = \sum_{j=1}^{n_f} J_{P,j}^T F_{ext,j}
  \]
\end{minipage}%
\begin{minipage}[t]{0.5\columnwidth}
  \textbf{Torques}
  \[
    \tau_{T_{ext}} = \sum_{k=1}^{n_m} J_{R,k}^T T_{ext,k}
  \]
\end{minipage}
\textbf{Actuators}
\[
  \tau_{a} = \sum_k (J_{S_k} - J_{S_{k-1}})^T F_{a,k} + (J_{R_k} -
  J_{R_{k-1}})^T T_{a,k}
\]

To compute $\tau$ (full EoM params):
\begin{enumerate}
  \item Use forward kinematics for Jacobians $J$.
  \item Compute $M(q)$ via Lagrange or PNE projection.
  \item Compute $b = C\dot{q}$ using Christoffel or recursive NE.
  \item Compute $g(q)$ from potential or NE gravity terms.
  \item Add external: $\tau = M\ddot{q} + b + g - J_c^T F_c$.
\end{enumerate}

\subsection{Velocity in Moving Bodies}

\textbf{Linear}
$^i v = ^i \dot{r} + ^i \omega \times ^i r$
\quad
\textbf{Angular} $^i \omega$
\quad
(Velocity in frame $i$)

\textbf{Twist vector}: $V =
\begin{bmatrix} v \\ \omega
\end{bmatrix}$

\textbf{Propagation}
$^i V_i = ^i A_{i-1} ^{i-1} V_{i-1} + ^i \dot{q}_i e_i$
\quad
($A$: adjoint).

\section{Dynamic Control}

\textbf{Control loops}\\
Position (inner velocity/torque),
Torque (feedforward dynamics).

\ssthl{Exam: Block diagrams for PD + gravity comp:}\\
$\tau = g(q) + K_p e + K_d \dot{e}$,
error $\ddot{e} + K_d \dot{e} + K_p e = M^{-1} \delta \tau$
Synchronize: Outer position loop with inner torque,\\
feedforward g for decoupling.

\subsection{Joint Impedance Control}

$\tau = g(q)
+ K_p (q_d - q)
+ K_d (\dot{q}_d - \dot{q})
+ K_i \int e \, dt + J^T F_{ext}$

Mass-spring-damper:
$\omega_n
= \sqrt{K_p / m}$, $\zeta
= K_d / (2 \sqrt{m K_p})$

\ssthl{Prove stability for PD control using Lyapunov Stability}\\
$V=\frac{1}{2}\dot{q}^T M \dot{q} + \frac{1}{2} e^T K_p e$,
$\dot{V} = -\dot{q}^T K_d \dot{q} \leq 0$\\
(LaSalle for convergence)

\subsection{Inverse Dynamics Control (Computed Torque)}

$\tau = M(q) (\ddot{q}_d + K_d \dot{e} + K_p e) + b(q,\dot{q}) + g(q)$

Decouples: $\ddot{e} + K_d \dot{e} + K_p e = 0$,
crit damp $K_d = 2 \sqrt{K_p}$

\subsection{Task-Space Dynamic Control}

\textbf{EoM}
\[
  \Lambda(x) \ddot{x} + \mu(x,\dot{x}) + p(x) = F + J^{-T} \tau_{ext}
\]
with $\Lambda = (J M^{-1} J^T)^{-1}$

\textbf{Control}
$F = \Lambda (\ddot{x}_d + K_d \dot{e}_x + K_p e_x) + \mu + p$

\textbf{Redundancy} weighted psd-inv:
$J^\dagger = W^{-1} J^T (J W^{-1} J^T)^{-1}$

\textbf{Null-space projector} $N = I - J^\dagger J$

\textbf{Multiple tasks}
Stack Jacobians, project secondary to $N$

\begin{itemize}
  \item Joint space: $\tau = M(q)(\ddot{q}^* + K_p e + K_d \dot{e}) +
    h(q, \dot{q})$ ($h = C\dot{q} + g$).
  \item With external force: $\tau_\text{total} = \tau_\text{motion}
    + J^T F_{ext}$.
  \item Task space: Solve $\ddot{q}_{des} = J^{\dagger}(\ddot{x}^* -
    \dot{J}\dot{q})$, then plug into EoM.
\end{itemize}

\subsection{End-Effector Dynamics}

As above and with feedforward $\ddot{x}_d$ from trajectory planning.

\ssthl{Exam: Hybrid with selection S (diag, 0=force,1=motion)}

\section{Interaction Control}

\subsection{Operational Space Control}
\begin{align*}
  \tau =
  &J^T \Lambda (\ddot{x}_d + K_d \dot{e}_x + K_p e_x - J M^{-1} (b + g))
  \\
  &+ (I - J^T \bar{J}^T) \tau_0\\
  \bar{J} =& \Lambda^{-1} J M^{-1}
\end{align*}

\ssthl{Exam: Formula for hybrid:
$\tau = J^T (S F_m + (I-S) F_f)$}

\subsection{Selection Matrix}

$S$: Diagonal matrix for hybrid control, e.g., $S_{ii}=1$ for
motion-controlled DOFs (position/velocity), $S_{ii}=0$ for
force-controlled DOFs. Allows blending impedance/force in task space
(e.g., force in z for peg-in-hole, position in x-y). Control law: $F
= S F_\text{motion} + (I-S) F_\text{force}$, where $F_\text{motion}$ uses PD,
$F_{force}$ is desired force.

\ssthl{Exam: Hybrid f-m via $S$ (e.g., $S=I$ motion, $S=0$ force)}

\subsection{Inverse Dynamics as QP}

\textbf{Formulation}
$\min_u \| A u - b \|^2_W$ s.t. constraints (torque limits,...)

\textbf{Hierarchical}
Solve primary, project secondary to null

\textbf{Hierarchical Optimization (QP) Formulation}\\
$\min_{\ddot{q}, \tau, f_c} \|\tau\|^2 + \epsilon \|\ddot{q}\|^2 +
\epsilon \|f_c\|^2$ s.t.\\
- Dynamics: $M \ddot{q} + h = S^T \tau + J_c^T f_c$.\\
- Contact: $J_c \ddot{q} + \dot{J}_c \dot{q} = 0$.\\
- Task (e.g., CoM): $J_{com} \ddot{q} + \dot{J}_{com} \dot{q} =
\ddot{x}_{com}^*$.\\
- Bounds: $|\tau| \leq \tau_{max}$, friction cone $\|f_{c,\perp}\|
\leq \mu f_{c,z}$.\\
\textbf{Matrix form}: Stack into $A x = b$ (e.g., $x = [\dot{u}^T,
\tau^T, F_c^T]^T$).

\textbf{Exampel} QP for 7DOF arm:
min $\| \dot{q} \|$ s.t. $J \dot{q} = \dot{x}_d$,
torque bounds; null for secondary (e.g., obstacle avoid)

\section{Floating Base Dynamics}

For mobile/legged robots: Unactuated base.

\subsection{Generalized coordinates}
\[
  q =
  \begin{pmatrix}
    q_b\\q_j
  \end{pmatrix}
  \quad\text{with}\quad
  q_b =
  \begin{pmatrix}
    q_{b_P}\\q_{b_R}
  \end{pmatrix}
  \ \in\
  \mathbb{R}^{3}\times SO(3)
\]

\subsection{Generalized Velocities/Accelerations}

Twist-based: $u = [^I v_b^T \, ^b \omega_b^T \, \dot{q}_j^T]^T \in
\mathbb{R}^{6+n_j}$; $\dot{u}$ similar. Map: $u = E_{fb} \dot{q}$
($E_{fb}$ handles rot param, e.g., quats to ang vel).

\ssthl{Note $\dot{q} \neq u$ due to SO(3) use for non-holonomic systems}

\subsection{Generalized velocities and accelerations}
\[
  u =
  \begin{pmatrix}
    _Iv_B\\
    _B\omega_IB\\
    \dot{\varphi}_1\\
    \vdots\\
    \dot{\varphi}_{n_j}\\
  \end{pmatrix}
  \quad \dot{u} =
  \begin{pmatrix}
    _Ia_B\\
    _B\psi_IB\\
    \ddot{\varphi}_1\\
    \vdots\\
    \ddot{\varphi}_{n_j}\\
  \end{pmatrix}
  \ \in\
  \mathbb{R}^{6+n_j}=
  \mathbb{R}^{n_u}
\]
\[
  u= E_{fb}\cdot\dot{q}
  \quad\text{with}\quad
  E_{fb}=
  \begin{bmatrix}
    1_{3\times3}&0&0\\
    0&E_{\chi_R}&0\\
    0&0&1_{n_j\times n_j}
  \end{bmatrix}
\]
$E_{fb}$ maps quaternions/Euler to twists.

Note: Slides often simplify to
$\mathbf{M}(\mathbf{q})\ddot{\mathbf{q}} +
\mathbf{h}(\mathbf{q},\dot{\mathbf{q}}) = \mathbf{S}^\top
\boldsymbol{\tau} + \mathbf{J}_c^\top \mathbf{F}_c$ for $\mathbf{h} =
\mathbf{b} + \mathbf{g}$. Important property: $\dot{\mathbf{M}} -
2\mathbf{C}$ is skew-symmetric (passivity for control proofs).

\subsection{Differential Kinematics}

Floating:
$J = [J_b \ J_j]$, $\dot{x} = J(q) u$
(task vel from gen. vel)

\subsection{Contacts and Constraints}

\textbf{Hard}
$J_c u = 0$ (no-slip)
\quad
\textbf{Const. acc.}
$J_c \dot{u} + \dot{J}_c u = 0$

\textbf{Soft}
$F_c = k \delta + d \dot{\delta}$
\quad
\textbf{Friction}
Cone $|F_t| \leq \mu F_n$

\ssthl{Exam: Enforce via multipliers $\lambda = -F_c$ \\
impacts $\Delta u = -(J_c M^{-1} J_c^T)^{-1} J_c u^-$}

\section{Dynamics of Floating Base Systems}

\begin{sstFullFrame}
  \color{white}
  \[
    \textbf{EoM}\quad
    M(q) \dot{u} + h(q,u) = S^T \tau + J_c^T F_c
  \]
\end{sstFullFrame}

where $S =
\begin{bmatrix} 0 & I
\end{bmatrix}$ (underactuated base),
$h = C u + g$.

Centroidal: $A_G \dot{u} + \dot{A}_G u = \sum F_{ext} + g_G$

(CMM $A_G$ for momentum).

\ssthl{Exam: Derive centroidal momentum for quadruped,\\
control CoM vel via $A_G u$ for balance under disturbances.}

\subsection{Constraint-Consistent Dynamics}

Project to null-space of constraints:
$\bar{M} \dot{\bar{u}} + \bar{h} = \bar{S}^T \tau$.

Full EoM: $M \dot{u} + h = S^T \tau + J_c^T F_c$.\\
Constraint: $J_c \dot{u} = -\dot{J}_c u$ (no slip).\\
Projected dynamics: Use projector $P$ ($P J_c^T = 0$): $P(M
\dot{u} + h) = P S^T \tau$. (Eliminates $F_c$ for feasibility checks.)

\subsection{Contact Dynamics}

\textbf{Impacts}
Instant velocitiy change:

$\Delta u = - (J_c M^{-1} J_c^T)^{-1} J_c u^- $
(pre-impact)

\textbf{Soft}
Spring-damper
$F_c = k \delta + d \dot{\delta}$

\subsection{Dynamic Control Methods}

\textbf{Multi-task}
CoM, feet as priorities.

Inv dyn: $\tau = S^+ (M \dot{u}_d + h - J_c^T F_{c,d}) + N \tau_0$

\ssthl{Exam: $\dot{u}_d$ from tasks; QP for torque opt w/ cones.}

\subsection{Lyapunov Stability for PD Control}

For $\tau = -K_p \tilde{q} - K_d \dot{q} + g(q)$:\\
Candidate $V =
\frac{1}{2} \dot{q}^T M \dot{q}
+ U_g + \frac{1}{2} \tilde{q}^T K_p \tilde{q}$,\\
$\dot{V} = -\dot{q}^T K_d \dot{q} \leq 0$
(asymptotically stable if $K_p, K_d > 0$)

