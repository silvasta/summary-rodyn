\section{Rotorcrafts}

\subsection{Key Assumptions and Modeling}
\begin{itemize}
  \item Core assumptions: CoG at body frame origin; rigid, symmetric
    structure; rigid propellers; neglect fuselage drag; near-hover
    (hub forces \& rolling moments $\approx 0$).
  \item Generalized coordinates for aerial robots:
    \[
      q = [C_{BW}, v,
      \omega, \alpha_i, \omega_i]^T \in \text{SO}(3) \times
      \mathbb{R}^{6+10}
    \]
    (incl. base orientation, velocities, arm angles, rotor speeds)
  \item Newton-Euler equations (full dynamics):
    \[
      \sum F_\text{ext}
      = m(\alpha_i, \omega_i) \, \dot{v}
      + b(v, \omega, \alpha_i, \omega_i)
      + g(C_{BW}, \alpha_i, \omega_i)
    \]
    \[
      \sum T_\text{ext}
      = I_B(\alpha_i, \omega_i) \, \dot{\omega}
      + b(v, \omega, \alpha_i, \omega_i)
      + g(C_{BW}, \alpha_i, \omega_i)
    \]
  \item Simplified for quadrotor: $q = [\Theta, v, \omega]^T \in
    \mathbb{R}^3 \times \mathbb{R}^6$.
  \item Derivation methods: Newton-Euler momentum theory, Lagrange,
    or virtual work principle.
\end{itemize}

\subsection{Propulsion and Thrust Models}
\begin{itemize}
  \item Thrust and drag torque (hover/low-speed):\\
    $T_i = b \omega_{p,i}^2$, $Q_i = d \omega_{p,i}^2$
  \item Classical momentum theory (ideal hover):
    \[
      T = 2 \rho A v_i^2, \quad P_{\text{ideal}} = T v_i =
      \frac{T^{3/2}}{\sqrt{2 \rho A}}.
    \]
  \item Figure of Merit: $FM = P_{\text{ideal}} / P_{\text{actual}} < 1$.
  \item Dependencies: Thrust $T \propto \omega^2$ (independent of
    forward speed near hover); drag torque $Q \propto \omega^2$.
\end{itemize}

\subsection{Control Allocation (Quadrotor, X-Configuration)}
\begin{itemize}
  \item Virtual inputs:
    \[
      \begin{aligned}
        u_1 &= b(\omega_1^2 + \omega_2^2 + \omega_3^2 + \omega_4^2) &
        &\text{(collective thrust)}, \\
        u_2 &= l b (\omega_4^2 - \omega_2^2) & &\text{(roll moment)}, \\
        u_3 &= l b (\omega_1^2 - \omega_3^2) & &\text{(pitch moment)}, \\
        u_4 &= d (-\omega_1^2 + \omega_2^2 - \omega_3^2 + \omega_4^2)
        & &\text{(yaw moment)}.
      \end{aligned}
    \]
  \item Allocation matrix (solve for $\omega_i^2$):
    \[
      \begin{bmatrix} \omega_1^2 \\ \omega_2^2 \\ \omega_3^2 \\ \omega_4^2
      \end{bmatrix}
      = \frac{1}{4}
      \begin{bmatrix}
        \frac{1}{b}  &  0         & \frac{1}{lb}  & -\frac{1}{d} \\
        \frac{1}{b}  & -\frac{1}{lb} &  0         &  \frac{1}{d} \\
        \frac{1}{b}  &  0         & -\frac{1}{lb} & -\frac{1}{d} \\
        \frac{1}{b}  &  \frac{1}{lb} &  0         &  \frac{1}{d}
      \end{bmatrix}
      \begin{bmatrix} u_1 \\ u_2 \\ u_3 \\ u_4
      \end{bmatrix}.
    \]
    Note: For + configuration, adjust signs
    (e.g., $u_2 = l b (\omega_1^2 + \omega_3^2 - \omega_2^2 - \omega_4^2))$
    exams often specify X-config.
  \item For over-actuated systems (e.g., hexacopter): Multiple
    solutions; choose minimum-energy (min rotor-speed norm) or min
    max($\omega_i$).
\end{itemize}

\subsection{Attitude Dynamics (Linearized near Hover)}
\begin{itemize}
  \item Linearized equations:
    \[
      \ddot{\phi} \approx \frac{u_2}{I_{xx}} = \frac{l b}{I_{xx}}
      (\omega_4^2 - \omega_2^2), \quad \ddot{\theta} \approx
      \frac{u_3}{I_{yy}}, \quad \ddot{\psi} \approx \frac{u_4}{I_{zz}}.
    \]
  \item PD attitude controller (decoupled double integrators):
    \[
      \begin{aligned}
        u_2 &= k_{p\phi}(\phi_d - \phi) - k_{d\phi} \dot{\phi}, \\
        u_3 &= k_{p\theta}(\theta_d - \theta) - k_{d\theta} \dot{\theta}, \\
        u_4 &= k_{p\psi}(\psi_d - \psi) - k_{d\psi} \dot{\psi}.
      \end{aligned}
    \]
  \item Tuning hint: $k_p$ sets rise time/natural frequency; $k_d$
    sets damping (avoid large $k_d$ to prevent saturation).
\end{itemize}

\subsection{Position and Altitude Control}
\begin{itemize}
  \item Altitude (small-angle approx.):
    \[
      \ddot{z} \approx g - u_1 m^{-1} \cos\phi \cos\theta
    \]
    \[
      u_1 = (m ( g - k_{p_z}(z_d - z) - k_{d_z} \dot{z}))
      (\cos\phi \cos\theta)^{-1}
    \]
  \item Thrust vector control (desired $^E\mathbf{T} = [T_x, T_y, T_z]^T$):
    \[
      u_1 = \| {^E\mathbf{T}} \|,
      \theta_d = \arcsin\left(\frac{T_x}{u_1}\right),
      \phi_d = -\arcsin\left(\frac{T_y}{u_1}\right)
    \]
  \item Forward acceleration: $\propto \sin\theta$ or $\sin\phi$
    (indep. of yaw $\psi$)
\end{itemize}

\subsubsection{Tricks and Hints}
\begin{itemize}
  \item Singularities: Euler angles singular at $\theta = \pm
    90^\circ$; use quaternions for aggressive maneuvers.
  \item Derivatives: Near hover, $\dot{\phi} \approx p$,
    $\dot{\theta} \approx q$, $\dot{\psi} \approx r$.
  \item Linearization: Small-angle approx. $\sin\phi \approx \phi$,
    $\cos\phi \approx 1$.
  \item Power: $\propto \omega^3$; minimize max($\omega_i$) over sum
    $\omega_i^2$.
  \item Underactuation: Position/attitude coupled; check config. (X
    vs. +) for allocation.
  \item Common pitfalls: Division by $\cos\phi \cos\theta$ (numerical
    issues); neglect drag in hover.
  \item Thrust vector: arcsin can cause singularities at high tilts;
    use quaternions for large angles (as in slides).
\end{itemize}

