
\section{Legged Robotics}

\subsection{Generalized Coordinates \& DOF Counting}
$\mathbf{q} =
\begin{bmatrix} \mathbf{q}_b \\ \mathbf{q}_j
\end{bmatrix}$ (dim: $6$ base + $n_j$ joints, e.g., 18 for 12-joint quad) \\
- Actuated: $\mathbf{q}_j$ (dim: $n_j$, controlled by torques
$\boldsymbol{\tau}$). \\
- Unactuated: $\mathbf{q}_b$ (floating base, 6 DOF). \\
- Contact constraints (point feet, no slip): $3$ per stance foot
(holonomic: $\mathbf{J}_c \dot{\mathbf{q}} = 0$). \\
\textbf{Exam trick:} Controllable DOF = total DOF $-$ constraints
(e.g., 3 stance legs: 18 $-$ 9 = 9; 3 for swing $\to$ 6 internal/force DOF). \\
- Underactuation degree: unactuated DOF $-$ constraints (if $>0$,
system underactuated).

\subsection{Dynamics Equations}
Core equation for floating-base system:
\[
  \mathbf{M}(\mathbf{q})\,\ddot{\mathbf{q}} +
  \mathbf{b}(\mathbf{q},\dot{\mathbf{q}}) + \mathbf{g}(\mathbf{q}) +
  \mathbf{J}_c^T \mathbf{F}_c = \mathbf{S}^T \boldsymbol{\tau}
\]
\textbf{Trick:} For support-consistent dynamics, project into
null-space of $\mathbf{J}_c$ (removes $\mathbf{F}_c$ dependency):
$\mathbf{N}_c = \mathbf{I} - \mathbf{J}_c^+ \mathbf{J}_c$.

\subsubsection{Centroidal Momentum (Balance Control)}

Centroidal momentum matrix $A_G(q)$:\\
$\dot{h}_G = A_G \dot{q}
= \sum m_i (J_{P,i}^T v_i + J_{R,i}^T (I_i \omega_i
+ \omega_i \times I_i \omega_i))$\\
CoM tasks: Control $\dot{h}_G$ via contacts (underactuated base)

\subsection{Contact Constraints}

- Velocity level: $\mathbf{J}_c \dot{\mathbf{q}} = 0$ (no motion at
contact points). \\
- Acceleration level: $\mathbf{J}_c \ddot{\mathbf{q}} +
\dot{\mathbf{J}}_c \dot{\mathbf{q}} = 0$. \\
- Friction: cone constraints on $\mathbf{F}_c$ (e.g.,
$\|\mathbf{F}_{c,\perp}\| \leq \mu \mathbf{F}_{c,\parallel}$). \\
\textbf{Exam hint:} Always enforce as highest priority in
hierarchical control to avoid slippage.

\subsection{Inverse Differential Kinematics (Swing Leg Task)}

Given constraints $\mathbf{J}_c \dot{\mathbf{q}} = 0$ and task
$\mathbf{J}_\text{swing} \dot{\mathbf{q}} =
\dot{\mathbf{r}}_\text{swing}^\text{des}$ \\
- Stacked (may be singular): $\dot{\mathbf{q}} =
\begin{bmatrix} \mathbf{J}_c \\ \mathbf{J}_\text{swing}
\end{bmatrix}^+
\begin{bmatrix} 0 \\ \dot{\mathbf{r}}_\text{swing}^\text{des}
\end{bmatrix}$ \\
- Hierarchical/null-space (preferred, robust):\\
\[
  \dot{\mathbf{q}} = \mathbf{J}_c^+ \cdot 0 + \mathbf{N}_c \dot{\mathbf{q}}_0
  \quad
  \dot{\mathbf{q}}_0 = (\mathbf{J}_\text{swing} \mathbf{N}_c)^+
  \dot{\mathbf{r}}_\text{swing}^\text{des}
\]
\textbf{Trick:} Use for tasks like swing foot tracking; extend to
acceleration level for dynamics.

\subsection{Control Strategies}
Common approaches (by robustness/exam frequency): \\
1. High-gain joint PD/PID: $\boldsymbol{\tau} = \mathbf{K}_p
(\mathbf{q}^* - \mathbf{q}) + \mathbf{K}_d (\dot{\mathbf{q}}^* -
\dot{\mathbf{q}})$ \\
(poor for impacts due to instability on uneven terrain,
prefer torque-based for compliance)\\
2. Inverse dynamics + low-gain: $\boldsymbol{\tau} =
\boldsymbol{\tau}_{FB} + \boldsymbol{\tau}_{FF}$ \\
(model-based feedforward) \\
3. Support-consistent ID: Project desired $\ddot{\mathbf{q}}^*$ into
$\mathbf{N}_c$ null-space. \\
4. Task-space control: Regulate tasks (e.g., CoM, feet) via QP. \\
5. Hierarchical QP: Strict priorities (dynamics/contact highest). \\
\textbf{Hint:} High-gain fails due to impacts/unmodeled terrain;
prefer torque control for compliance.

\subsection{Actuator Types}
\begin{tabular}{l|c|c}
  Type  & Torque Method & Pros/Cons  \\
  \hline
  High-gear +   & Elasticity/sensor & Robust but \\
  SEA/sensor  &  &   slower response\\
  \hline
  Low-gear + & Current $\approx$ torque & Fast, backdrivable  \\
  current  &  &   impact-resistant \\
  \hline
  Hydraulic  & Pressure valve & High power, hard to scale \\
\end{tabular}
\textbf{Trick:} Low-gear best for dynamic locomotion (e.g.Mini Cheetah)

\subsection{Whole-Body Control (WBC) Hierarchy}

Typical priority order: \\
1. Dynamics: $\mathbf{M} \ddot{\mathbf{q}} + \mathbf{h} =
\mathbf{J}_c^T \mathbf{f}_c + \mathbf{S}^T \boldsymbol{\tau}$. \\
2. Contact stability: $\mathbf{J}_c \ddot{\mathbf{q}} +
\dot{\mathbf{J}}_c \dot{\mathbf{q}} = 0$. \\
3. Tasks (CoM/base/swing): e.g., $\mathbf{J}_{com} \ddot{\mathbf{q}}
= \ddot{\mathbf{x}}_{com}^{ref} - \dot{\mathbf{J}}_{com} \dot{\mathbf{q}}$. \\
4. Regularization: $\min \|\boldsymbol{\tau}\|$ or $\|\mathbf{f}_c\|$. \\
Torque command:
\[
  \boldsymbol{\tau}^{d} =
  \mathbf{M}_j(\mathbf{q})\ddot{\mathbf{q}}^{*}
  + \mathbf{h}_j(\mathbf{q},\dot{\mathbf{q}})
  - \mathbf{J}_{s,j}^{T}(\mathbf{q})\boldsymbol{\lambda}^{*}
\]
Final: $\boldsymbol{\tau}^{ref} = \boldsymbol{\tau}^{d} + k_p
\tilde{\mathbf{q}} + k_d \dot{\tilde{\mathbf{q}}}$.

\subsection{Optimization-Based Control (QP/SLQ)}
QP for WBC:
$\min \|\mathbf{A} \mathbf{x} - \mathbf{b}\|^2$
s.t. constraints (tasks, limits) \\
Finite-time OCP (SLQ/DDP):
\[
  \min_{u(\cdot)} \Phi(x(T)) + \int_{0}^{T} L(x(t),u(t),t)\,dt \quad
  \text{s.t. constraints}
\]
\textbf{Trick:} SLQ linear in horizon length (faster than DDP); use
for MPC in legged systems.

\subsection{Exam Tricks \& Hints}
- Why torque control? Compliance for rough terrain (vs. stiff
position control). \\
- Underactuation: Torque can't directly control base; use contacts. \\
- Learning (RL): Robust to sim2real gaps in contacts/perception. \\
- Common Q: Derive controllable DOF for given stance; formulate QP
for multi-task control.

