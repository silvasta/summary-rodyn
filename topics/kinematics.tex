\section{Vectors and Positions}

Position vectors:
Parametrize $P$ in frame $\mathcal{A}$ as
${}_\mathcal{A}\mathbf{r}_{AP} = \mathbf{r}(\chi)$,\\
where $\chi$ are parameters (e.g., Cartesian coords)

\subsection{Linear Velocity}

\[
  r        = r(\chi)
  \quad\quad
  \dot{r}  = \frac{\partial r}{\partial \chi}\dot{\chi}
  = E_p(\chi)\dot{\chi}
\]

\ssthl{Exam tip: Used in Jacobians for task-space velocities}

\subsection{Rotations}

Rotation matrix \( \mathbf{C}_{\mathcal{A}\mathcal{B}} \) transforms
vectors from frame \( \mathcal{B} \) to \( \mathcal{A} \):
\[
  {}_\mathcal{A}\mathbf{r}_{AP} = \mathbf{C}_{\mathcal{A}\mathcal{B}}
  \cdot {}_\mathcal{B}\mathbf{r}_{AP}
\]
\textbf{Properties}
Orthogonal (\( \mathbf{C}^T = \mathbf{C}^{-1} \)),
det=1 for proper rotations

\textbf{Elementary rotations} (about x,y,z axes by angle \( \theta \)):
\[
  \mathbf{R}_x(\theta) =
  \begin{bmatrix} 1 & 0 & 0 \\ 0 & \cos\theta & -\sin\theta \\ 0 &
    \sin\theta & \cos\theta
  \end{bmatrix}
\]
\[
  \mathbf{R}_y(\theta) =
  \begin{bmatrix} \cos\theta & 0 & \sin\theta \\ 0 & 1 & 0
    \\ -\sin\theta & 0 & \cos\theta
  \end{bmatrix}
\]
\[
  \mathbf{R}_z(\theta) =
  \begin{bmatrix} \cos\theta & -\sin\theta & 0 \\ \sin\theta &
    \cos\theta & 0 \\ 0 & 0 & 1
  \end{bmatrix}
\]

\textbf{Composition} \( \mathbf{C}_{\mathcal{A}\mathcal{C}} =
  \mathbf{C}_{\mathcal{A}\mathcal{B}} \cdot
\mathbf{C}_{\mathcal{B}\mathcal{C}} \).

\textbf{Homogeneous transformations} (4x4 for position + orientation)
\[
  \mathbf{T}_{\mathcal{A}\mathcal{B}} =
  \begin{bmatrix} \mathbf{C}_{\mathcal{A}\mathcal{B}} &
    {}_\mathcal{A}\mathbf{r}_{AB} \\ \mathbf{0}_{1\times3} & 1
  \end{bmatrix}
\]
\textbf{Passive} Rotate frame\quad \textbf{Active} Rotate vector

\ssthl{Exam pitfall: Distinguish for inverse problems}

\subsection{Angular Velocity}

Angular velocity \( \boldsymbol{\omega}\) satisfies
\( \dot{C} = \boldsymbol{\omega} \times \mathbf{C} \),
\( \dot{C} = \mathbf{S}(\boldsymbol{\omega}) \mathbf{C} \)
\[
  \mathbf{S}(\mathbf{a}) =
  \begin{bmatrix}
    0 & -a_z & a_y \\ a_z & 0 & -a_x \\ -a_y & a_x & 0
  \end{bmatrix}
\]

\textbf{Composition:}
\( {}_\mathcal{A}\boldsymbol{\omega}_{\mathcal{A}\mathcal{C}} =
  {}_\mathcal{A}\boldsymbol{\omega}_{\mathcal{A}\mathcal{B}} +
  \mathbf{C}_{\mathcal{A}\mathcal{B}}
{}_\mathcal{B}\boldsymbol{\omega}_{\mathcal{B}\mathcal{C}} \).

% WARN: Notation recap L2.3

\subsection{Parametrization of 3d Rotations}

\textbf{Minimal params}: 3 (due to SO(3) manifold)\\
Common for avoiding singularities in kinematics/control.

\begin{itemize}[itemsep=0pt]
  \item \textbf{Rotation matrix} 9 params, 6 orthonormality constraints.
    Direct but redundant.
  \item \textbf{Euler angles} (e.g., ZYZ):
    \( \mathbf{C} = \mathbf{R}_z(\phi)
    \mathbf{R}_y(\theta) \mathbf{R}_z(\psi) \)\\
    3 params, \textbf{singularities} at \( \theta = 0,\pi \) (gimbal lock).\\
    \ssthl{Exam: Derive matrix; convert to/from}
  \item \textbf{Angle-axis}
    No singularities but multi-valued.\\
    \( \mathbf{C} = \exp(\mathbf{S}(\mathbf{k}\theta)) = \mathbf{I} +
      \sin\theta \mathbf{S}(\mathbf{k}) + (1-\cos\theta)
    \mathbf{S}^2(\mathbf{k}) \)
    (Rodrigues) \( \mathbf{k} \) unit vector, \( \theta \) angle.
  \item \textbf{Rotation vector}
    \( \boldsymbol{\rho} = \mathbf{k}\theta \),
    Similar to angle-axis.
  \item \textbf{Unit quaternions}
    4 params, 1 constraint.
    No singularities; efficient for interpolation/composition.
    \\    \( \mathbf{q} = (q_0, \mathbf{q}_v) = (\cos(\theta/2), \mathbf{k}
    \sin(\theta/2)) \), \( \|\mathbf{q}\|=1 \).
\end{itemize}

\subsection{Unit Quaternions}
\textbf{To rotation matrix}
\[
  \mathbf{C}(\mathbf{q}) = \mathbf{I} + 2q_0 \mathbf{S}(\mathbf{q}_v)
  + 2\mathbf{S}^2(\mathbf{q}_v)
\]
\textbf{From matrix}
Extract \( \theta = \arccos((\text{trace}(\mathbf{C})-1)/2) \)
\[
  \mathbf{q}_1 \circ \mathbf{q}_2 = (q_{10}q_{20} -
    \mathbf{q}_{1v} \cdot \mathbf{q}_{2v}, q_{10}\mathbf{q}_{2v} +
  q_{20}\mathbf{q}_{1v} + \mathbf{q}_{1v} \times \mathbf{q}_{2v})
\]
\textbf{Rotate vector}
\( \mathbf{v}' = \mathbf{q} \circ (0,\mathbf{v}) \circ
\mathbf{q}^{-1} \) (pure quaternion)

\textbf{Time derivative}
\( \dot{\mathbf{q}} = \frac{1}{2} \mathbf{q} \circ (0, \boldsymbol{\omega}) \)

\ssthl{Exam: Use for singularity-free velocity integration.}

\section{Multi Body Kinematics}

\textbf{Generalized coordinates}
Joint variables \( \mathbf{q} = (q_1, \dots, q_n)^T \)\\
(e.g., angles for revolute, displacements for prismatic)

\textbf{End-effector} configuration
\( \boldsymbol{\chi}_e = (\boldsymbol{\chi}_{eP}, \boldsymbol{\chi}_{eR})^T \)\\
(position + orientation params)

\textbf{Operational/task space}
Subset \( \boldsymbol{\chi}_o \) for specific tasks \\
(e.g., position only)

\subsection{Forward Kinematics}

% HACK: L3.13 picture!

End-effector configuration \( \boldsymbol{\chi}_e = f(\mathbf{q}) \).
For serial chains: Product of homogeneous transforms
\( \mathbf{T}_{0n} = \mathbf{T}_{01} \mathbf{T}_{12} \cdots
\mathbf{T}_{(n-1)n} \)

\textbf{Denavit-Hartenberg (DH) params}\\
Standard for link modeling \ssthl{(crucial for exams!)}\\

Transform $\mathbf{T}_{i-1,i} =$
\[
  \begin{bmatrix} \cos\theta_i & -\sin\theta_i \cos\alpha_i &
    \sin\theta_i \sin\alpha_i & a_i \cos\theta_i \\ \sin\theta_i &
    \cos\theta_i \cos\alpha_i & -\cos\theta_i \sin\alpha_i & a_i
    \sin\theta_i \\ 0 & \sin\alpha_i & \cos\alpha_i & d_i \\ 0 & 0 & 0 & 1
  \end{bmatrix}
\]
Link length: \( a_i \)|
Link twist: \( \alpha_i \)|
Link offset: \( d_i \)|
Joint angle: \( \theta_i \)

\ssthl{Exam: Assign DH table for given robot;
compute forward map; analyze workspace}

\subsection{Jacobians}

\begin{sstTitleBox}[ForestGreen]{
    Jacobi-Box
  }
  \begin{sstFullFrame}[ForestGreen]
    \color{white}
    \[
      \text{Differential map:} \quad
      \dot{\chi_e} = J_{eA}(q)\dot{q}
    \]
    % TODO: why the A in J_{eA}
    \[
      \text{Differential map:} \quad
      \dot{\boldsymbol{\chi}}_e = \mathbf{J}_{eA}(\mathbf{q}) \dot{\mathbf{q}}
    \]
  \end{sstFullFrame}
  \begin{sstOnlyFrame}[ForestGreen]
    \[
      J_{eA} =
      \frac{\partial \chi_e}{\partial q} =
      \begin{bmatrix}
        \frac{\partial \chi_1}{\partial q_1}
        & \dots  &
        \frac{\partial \chi_1}{\partial q_n}
        \\
        \vdots & \ddots & \vdots
        \\
        \frac{\partial \chi_m}{\partial q_1}
        & \dots  &
        \frac{\partial \chi_m}{\partial q_n}
      \end{bmatrix}
    \]
  \end{sstOnlyFrame}
  \begin{sstOnlyFrame}[ForestGreen]
    \textbf{Analytical Jacobian}
    For orientation params (Euler rates,..)
  \end{sstOnlyFrame}
  \begin{sstOnlyFrame}[ForestGreen]
    \textbf{Geometric Jacobian}
    Columns from velocity contributions\\
    (prismatic: linear velocity;
      revolute:
    \( \boldsymbol{\omega}_i \times \mathbf{r}_{i e} + \mathbf{v}_i \)).
  \end{sstOnlyFrame}
  \begin{sstOnlyFrame}[ForestGreen]
    \textbf{Prismatic}
    Position
    \( \mathbf{J}_{P,i} = \mathbf{z}_{i-1} \),
    Rotation
    \( \mathbf{J}_{O,i} = \mathbf{0} \)

    \textbf{Revolute}
    Pos.
    \( \mathbf{J}_{P,i} = \mathbf{z}_{i-1} \times \mathbf{r}_{i e} \),
    Rotation
    \( \mathbf{J}_{O,i} = \mathbf{z}_{i-1} \)
  \end{sstOnlyFrame}
  \begin{sstOnlyFrame}[ForestGreen]
    \textbf{Singularity} $det(J)=0 \to$ loss of DOF

    \ssthl{Exam: Compute rank, manipulability}
    \( \mu = \sqrt{\det(\mathbf{J}\mathbf{J}^T)} \).
  \end{sstOnlyFrame}
\end{sstTitleBox}

\subsection{Velocity in Moving Bodies}

\textbf{Rigid Body Formulation}
Point P velocity in frame A:\\
\( {}_\mathcal{A}\mathbf{v}_P =
  {}_\mathcal{A}\mathbf{v}_B +
  {}_\mathcal{A}\boldsymbol{\omega}_{\mathcal{A}\mathcal{B}} \times
  {}_\mathcal{A}\mathbf{r}_{BP} + \mathbf{C}_{\mathcal{A}\mathcal{B}}
{}_\mathcal{B}\mathbf{v}_{BP}^{\text{rel}} \).

\textbf{Twist vector}
\( \mathbf{t} = (\boldsymbol{\omega}, \mathbf{v})^T \),
propagates via adjoint matrices.

\ssthl{Exam: Recursive computation for serial chains (forward for velocities).}

\section{Inverse Kinematics}
\[
  w_e^\star = J_{e0} \dot{q}
\]
% TODO: better formulation
Solve \( \mathbf{q} = f^{-1}(\boldsymbol{\chi}_e^\star) \)

Analytical for low DOF (e.g., 3R planar: geometric)

Numerical otherwise (e.g., Newton-Raphson)

\textbf{Velocity level}
\( \dot{\mathbf{q}} = \mathbf{J}^\dagger \dot{\boldsymbol{\chi}}_e +
(\mathbf{I} - \mathbf{J}^\dagger \mathbf{J}) \dot{\mathbf{q}}_0 \)
\\(pseudo-inverse for redundancy; nullspace for secondary tasks)

\textbf{Singularities}
Use damped least-squares\\
\( \mathbf{J}^\dagger
= \mathbf{J}^T (\mathbf{J}\mathbf{J}^T + \lambda^2 \mathbf{I})^{-1} \)

\textbf{Redundancy}
If $n > m$, infinite solutions; optimize (e.g., min norm velocity).

\subsection{Multi-task control}

Single task: \( \dot{\mathbf{q}} = \mathbf{J}^\dagger
\dot{\boldsymbol{\chi}}^\star \).

Stacked: Combine Jacobians \( \mathbf{J}_s =
  \begin{bmatrix} \mathbf{J}_1 \\ \mathbf{J}_2
\end{bmatrix} \), solve if consistent.

Prioritized: \( \dot{\mathbf{q}} = \mathbf{J}_1^\dagger
  \dot{\boldsymbol{\chi}}_1^\star + (\mathbf{I} - \mathbf{J}_1^\dagger
  \mathbf{J}_1) \mathbf{J}_2^\dagger (\dot{\boldsymbol{\chi}}_2^\star -
\mathbf{J}_2 \mathbf{J}_1^\dagger \dot{\boldsymbol{\chi}}_1^\star) \).

\subsubsection{Error analysis}

Task error:
\( \mathbf{e} = \boldsymbol{\chi}^\star - \boldsymbol{\chi} \)\\
Control: \( \dot{\boldsymbol{\chi}}^\star = \dot{\boldsymbol{\chi}}_d
+ \mathbf{K} \mathbf{e} \)
(resolved rate)

\textbf{Joint trajectory}
Interpolate \( \mathbf{q}(t) \)
(e.g., cubic polynomial for vel/acc constraints)

\ssthl{Exam patterns: Compute inverse for 2-3 DOF arm; handle
redundancy/singularities in control loops.}

