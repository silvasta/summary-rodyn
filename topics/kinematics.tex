
\ssthl{Exam strategy:
  30\% DH/forward,
  40\% Jacobians/singularities,
  30\% inverse/control.
  Practice 3R examples;
link to dynamics (e.g., Jacobian in torque control).}

\section{Vectors and Positions}

Position vectors:
Parametrize $P$ in frame $\mathcal{A}$ as
${}_\mathcal{A}\mathbf{r}_{AP} = \mathbf{r}(\chi)$,\\
where $\chi$ are parameters (e.g., Cartesian coords)

\subsection{Linear Velocity}

\[
  r        = r(\chi)
  \quad\quad
  \dot{r}  = \frac{\partial r}{\partial \chi}\dot{\chi}
  = E_p(\chi)\dot{\chi}
\]

\ssthl{Exam tip: Basis for Jacobians\\
Compute for end-effector task velocities in control problems.}

\subsection{Rotations}

\textbf{Rotation matrix}
$\mathbf{C}_{\mathcal{A}\mathcal{B}}$ transforms
vectors from frame \( \mathcal{B} \) to \( \mathcal{A} \):
\[
  {}_\mathcal{A}\mathbf{r}_{AP} = \mathbf{C}_{\mathcal{A}\mathcal{B}}
  \cdot {}_\mathcal{B}\mathbf{r}_{AP}
\]
\textbf{Properties}
Orthogonal (\( \mathbf{C}^T = \mathbf{C}^{-1} \)),
det=1 for proper rotations

\textbf{Elementary rotations} (about x,y,z axes by angle \( \theta \)):
\[
  \mathbf{R}_x(\theta) =
  \begin{bmatrix} 1 & 0 & 0 \\ 0 & \cos\theta & -\sin\theta \\ 0 &
    \sin\theta & \cos\theta
  \end{bmatrix}
\]
\[
  \mathbf{R}_y(\theta) =
  \begin{bmatrix} \cos\theta & 0 & \sin\theta \\ 0 & 1 & 0
    \\ -\sin\theta & 0 & \cos\theta
  \end{bmatrix}
\]
\[
  \mathbf{R}_z(\theta) =
  \begin{bmatrix} \cos\theta & -\sin\theta & 0 \\ \sin\theta &
    \cos\theta & 0 \\ 0 & 0 & 1
  \end{bmatrix}
\]

\textbf{Composition}:
$\mathbf{C}_{\mathcal{A}\mathcal{C}}
= \mathbf{C}_{\mathcal{A}\mathcal{B}} \cdot \mathbf{C}_{\mathcal{B}\mathcal{C}}$

\textbf{Homogeneous transformations} (4x4 for position + orientation):
\[
  \mathbf{T}_{\mathcal{A}\mathcal{B}} =
  \begin{bmatrix} \mathbf{C}_{\mathcal{A}\mathcal{B}} &
    {}_\mathcal{A}\mathbf{r}_{AB} \\ \mathbf{0}_{1\times3} & 1
  \end{bmatrix}
\]
\textbf{Passive}: Rotate frame\quad
\textbf{Active}: Rotate vector

\ssthl{Exam pitfall:
Confuse active/passive in inverse kinematics, always specify frames.}

\subsection{Angular Velocity}

Angular velocity $\boldsymbol{\omega}$ satisfies
$\dot{\mathbf{C}} = \boldsymbol{\omega} \times \mathbf{C}$,
or $\dot{\mathbf{C}} = \mathbf{S}(\boldsymbol{\omega}) \mathbf{C}$
\[
  \mathbf{S}(\mathbf{a}) =
  \begin{bmatrix} 0 & -a_z & a_y \\ a_z & 0 & -a_x \\ -a_y & a_x & 0
  \end{bmatrix}.
\]

\textbf{Composition}:
${}^\mathcal{A}\boldsymbol{\omega}_{\mathcal{A}\mathcal{C}} =
{}^\mathcal{A}\boldsymbol{\omega}_{\mathcal{A}\mathcal{B}} +
\mathbf{C}_{\mathcal{A}\mathcal{B}}
{}^\mathcal{B}\boldsymbol{\omega}_{\mathcal{B}\mathcal{C}}$

% HACK: Notation recap L2.3

\ssthl{Exam note:
Use skew-symmetric for deriving velocity Jacobians}

\subsection{Parametrization of 3d Rotations}

\textbf{Minimal params}: 3 (due to SO(3) manifold)\\
Common for avoiding singularities in kinematics/control.

\begin{itemize}[itemsep=0pt]
  \item \textbf{Rotation matrix} 9 params, 6 orthonormality constraints.
    Direct but redundant.
  \item \textbf{Euler angles} (e.g., ZYZ):
    $\mathbf{C} = \mathbf{R}_z(\phi) \mathbf{R}_y(\theta) \mathbf{R}_z(\psi)$\\
    3 params, \textbf{singularities} at \( \theta = 0,\pi \) (gimbal lock).\\
    \ssthl{Exam: Derive matrix; convert to/from quaternions.}
  \item \textbf{Angle-axis}
    No singularities but multi-valued.\\
    $\mathbf{C} = \exp(\mathbf{S}(\mathbf{k}\theta))
    = \mathbf{I} + \sin\theta \mathbf{S}(\mathbf{k})
    + (1-\cos\theta) \mathbf{S}^2(\mathbf{k})$ (Rodrigues).
    $\mathbf{k}$ unit vector, $\theta$ angle.
  \item \textbf{Rotation vector}
    $\boldsymbol{\rho} = \mathbf{k}\theta$,
    Similar to angle-axis.
  \item \textbf{Unit quaternions}
    4 params, 1 constraint.
    No singularities, efficient for interpolation/composition. \\
    $\mathbf{q} = (q_0, \mathbf{q}_v)
    = (\cos(\theta/2), \mathbf{k} \sin(\theta/2))$, $\|\mathbf{q}\|=1$.
\end{itemize}

\subsection{Unit Quaternions}

\textbf{To rotation matrix}:
\[
  \mathbf{C}(\mathbf{q}) = \mathbf{I} + 2 \mathbf{S}^2(\mathbf{q}_v)
  + 2 q_0 \mathbf{S}(\mathbf{q}_v).
\]
\textbf{From matrix}
Extract $\theta = \arccos((\text{trace}(\mathbf{C})-1)/2)$
\[
  \mathbf{q}_1 \circ \mathbf{q}_2 = (q_{10}q_{20} -
    \mathbf{q}_{1v} \cdot \mathbf{q}_{2v}, q_{10}\mathbf{q}_{2v} +
  q_{20}\mathbf{q}_{1v} + \mathbf{q}_{1v} \times \mathbf{q}_{2v})
\]

\textbf{Rotate vector}
$\mathbf{v}' = \mathbf{q} \circ (0,\mathbf{v}) \circ \mathbf{q}^{-1}$
(pure quaternion)

\textbf{Time derivative}
$\dot{\mathbf{q}} = \frac{1}{2} \mathbf{q} \circ (0, \boldsymbol{\omega})$

\ssthl{Exam: Use for singularity-free integration,
common in control velocity loops.}

\section{Multi Body Kinematics}

\textbf{Generalized coordinates}
Joint variables
$\mathbf{q} = (q_1, \dots, q_n)^T$
(angles for revolute, displacements for prismatic).

\textbf{End-effector configuration}
$\boldsymbol{\chi}_e = (\boldsymbol{\chi}_{eP}, \boldsymbol{\chi}_{eR})^T$
\\(position + orientation params)

\textbf{Operational/task space}
Subset $\boldsymbol{\chi}_o$ for specific tasks \\
(e.g., position only)

\subsection{Forward Kinematics}

% HACK: L3.13 picture!

End-effector configuration $\boldsymbol{\chi}_e = f(\mathbf{q})$.
For serial chains: Product of homogeneous transforms
$\mathbf{T}_{0n} = \mathbf{T}_{01} \mathbf{T}_{12} \cdots
\mathbf{T}_{(n-1)n}$

\textbf{Denavit-Hartenberg (DH) params}
Standard for link modeling

\ssthl{Exam hint: For revolute, $\theta_i$ variable; prismatic, $d_i$
variable. Common pitfall: Wrong $x_i$ alignment—check perpendicularity.}

Transform $\mathbf{T}_{i-1,i} =$
\[
  \begin{bmatrix} \cos\theta_i & -\sin\theta_i \cos\alpha_i &
    \sin\theta_i \sin\alpha_i & a_i \cos\theta_i \\ \sin\theta_i &
    \cos\theta_i \cos\alpha_i & -\cos\theta_i \sin\alpha_i & a_i
    \sin\theta_i \\ 0 & \sin\alpha_i & \cos\alpha_i & d_i \\ 0 & 0 & 0 & 1
  \end{bmatrix}
\]
Link length: \( a_i \)|
Link twist: \( \alpha_i \)|
Link offset: \( d_i \)|
Joint angle: \( \theta_i \)

\textbf{Rules}
Align $z_i$ with joint axis $i+1$;
$x_i$ perpendicular to $z_{i-1}$ and $z_i$;
origin at intersection.

\ssthl{Exam: Assign DH for 3-6 DOF arms
  (e.g., SCARA, PUMA); compute pose;
analyze reachable workspace (volume, boundaries).}

\subsection{Workspace Analysis}

\textbf{Reachable}: All positions end-effector can reach (ignore
orientation). \textbf{Dexterous}: All poses.

For planar 2R: Annulus with radii $|l_1 - l_2|$ to $l_1 + l_2$. For
3R: Adds redundancy for orientation.

\ssthl{Exam pattern: Sketch workspace for given arm; identify
voids/holes due to joint limits.}

\subsection{Jacobians}

\begin{sstTitleBox}[ForestGreen]{
    Jacobi-Box
  }
  \begin{sstFullFrame}[ForestGreen]
    \color{white}
    \[
      \text{Differential map:} \quad
      \dot{\chi_e} = J_{eA}(q)\dot{q}
    \]
    % TODO: why the A in J_{eA}
    \[
      \text{Differential map:} \quad
      \dot{\boldsymbol{\chi}}_e = \mathbf{J}_{eA}(\mathbf{q}) \dot{\mathbf{q}}
    \]
  \end{sstFullFrame}
  \begin{sstOnlyFrame}[ForestGreen]
    \[
      J_{eA} =
      \frac{\partial \chi_e}{\partial q} =
      \begin{bmatrix}
        \frac{\partial \chi_1}{\partial q_1}
        & \dots  &
        \frac{\partial \chi_1}{\partial q_n}
        \\
        \vdots & \ddots & \vdots
        \\
        \frac{\partial \chi_m}{\partial q_1}
        & \dots  &
        \frac{\partial \chi_m}{\partial q_n}
      \end{bmatrix}
    \]
  \end{sstOnlyFrame}
  \begin{sstOnlyFrame}[ForestGreen]
    \textbf{Analytical Jacobian}
    For orientation params (Euler rates,..)
  \end{sstOnlyFrame}
  \begin{sstOnlyFrame}[ForestGreen]
    \textbf{Geometric Jacobian}
    Direct velocity map $\mathbf{t} = \mathbf{J}_G \dot{\mathbf{q}}$
    \\differs in orientation (angular velocity vs. rates)
  \end{sstOnlyFrame}
  \begin{sstOnlyFrame}[ForestGreen]
    \textbf{Prismatic}
    Position
    \( \mathbf{J}_{P,i} = \mathbf{z}_{i-1} \),
    Rotation
    \( \mathbf{J}_{O,i} = \mathbf{0} \)

    \textbf{Revolute}
    Pos.
    \( \mathbf{J}_{P,i} = \mathbf{z}_{i-1} \times \mathbf{r}_{i e} \),
    Rotation
    \( \mathbf{J}_{O,i} = \mathbf{z}_{i-1} \)
  \end{sstOnlyFrame}
  \begin{sstOnlyFrame}[ForestGreen]
    \textbf{Singularity}: $\det(\mathbf{J})=0 \to$ DOF loss. Types:
    Boundary (workspace edge), internal (e.g., aligned links).

    Manipulability: $\mu = \sqrt{\det(\mathbf{J}\mathbf{J}^T)}$;
    ellipsoid for velocity/force transmission.
  \end{sstOnlyFrame}
  \begin{sstOnlyFrame}[ForestGreen]
    Real-world tip:
    Use manipulability index for path planning;
    damp near singularities ($\lambda \propto 1/\mu$) to prevent instability.
  \end{sstOnlyFrame}
\end{sstTitleBox}

\ssthl{Exam: Compute $\mathbf{J}$ for 2-4 DOF,
  find singularities (e.g., $\theta_2=0$ in 3R),
condition number $\kappa = \sigma_{\max}/\sigma_{\min}$.}

\ssthl{Exam tip: For 3R planar (pos only), $J =$}
% TODO: explain, verify or correct
\[
  \begin{bmatrix} -l_1 s_1 - l_2 s_{12} - l_3 s_{123} & -l_2 s_{12} -
    l_3 s_{123} & -l_3 s_{123} \\ l_1 c_1 + l_2 c_{12} + l_3 c_{123}
    & l_2 c_{12} + l_3 c_{123} & l_3 c_{123}
  \end{bmatrix}
\]\ssthl{det=0 when collinear}

\subsection{Velocity in Moving Bodies}

\textbf{Rigid Body Formulation}
Point P velocity in frame A:\\
\[
  {}^\mathcal{A}\mathbf{v}_P =
  {}^\mathcal{A}\mathbf{v}_B +
  {}^\mathcal{A}\boldsymbol{\omega}_{\mathcal{A}\mathcal{B}} \times
  {}^\mathcal{A}\mathbf{r}_{BP} + \mathbf{C}_{\mathcal{A}\mathcal{B}}
  {}^\mathcal{B}\mathbf{v}_{BP}^{\text{rel}}
\]
\textbf{Twist vector}
$\mathbf{t} = (\boldsymbol{\omega}, \mathbf{v})^T$
propagates via adjoint:
\[
  \mathbf{t}_i =
  % WARNING: what is Ad?
  % TODO: write this clear
  \operatorname{Ad}_{\mathbf{T}_{i-1,i}} \mathbf{t}_{i-1} +
  \mathbf{e}_i \dot{q}_i
  \quad \text{with unit twist: } \mathbf{e}_i
\]
\ssthl{Exam: Recursive forward vel. for chains,
links to Newton-Euler dynamics}

\section{Inverse Kinematics}
\begin{sstFullFrame}
  \color{white}
  \[
    \textbf{Main idea: Solve}\quad
    \mathbf{q} = f^{-1}(\boldsymbol{\chi}_e^\star)
  \]
\end{sstFullFrame}

\textbf{Analytical} for low DOF, e.g. for 2R planar:

$\theta_2 = \pm \arccos((x^2 + y^2 - l_1^2 - l_2^2)/(2 l_1 l_2))$,

$\theta_1 =
\arctan(y,x) - \arctan(l_2 \sin\theta_2, l_1 + l_2 \cos\theta_2)$)

For 6R: Decouple position/orientation (spherical wrist)

\textbf{Numerical} Newton-Raphson:
$\mathbf{q}_{k+1} =
\mathbf{q}_k + \mathbf{J}^{-1} (\boldsymbol{\chi}^\star - f(\mathbf{q}_k))$

\textbf{Velocity level}\quad
$\dot{\mathbf{q}}
= \mathbf{J}^\dagger \dot{\boldsymbol{\chi}}_e
+ (\mathbf{I} - \mathbf{J}^\dagger \mathbf{J}) \dot{\mathbf{q}}_0$
\\(pseudo-inverse for redundancy, nullspace optimization)

\textbf{Singularities}
Damped least-squares: $\mathbf{J}^\dagger = \mathbf{J}^T
(\mathbf{J}\mathbf{J}^T + \lambda^2 \mathbf{I})^{-1}$

\textbf{Redundancy}: $n > m$; min-norm or secondary tasks\\
(e.g., joint limit avoidance)

\ssthl{Exam: Solve inverse for 3R arm,
handle multiple solutions/elbow configs.}

\ssthl{Exam pattern: For PUMA-like (spherical wrist),
solve position first (joints 1-3),
then orientation (4-6); multiple wrist configs.}

\subsection{Multi-task control}

Single task:
\( \dot{\mathbf{q}} = \mathbf{J}^\dagger
\dot{\boldsymbol{\chi}}^\star \).

Stacked: Combine Jacobians \( \mathbf{J}_s =
\begin{bmatrix} \mathbf{J}_1 \\ \mathbf{J}_2
\end{bmatrix} \), solve if consistent.

Prioritized: \( \dot{\mathbf{q}} = \mathbf{J}_1^\dagger
\dot{\boldsymbol{\chi}}_1^\star + (\mathbf{I} - \mathbf{J}_1^\dagger
\mathbf{J}_1) \mathbf{J}_2^\dagger (\dot{\boldsymbol{\chi}}_2^\star -
\mathbf{J}_2 \mathbf{J}_1^\dagger \dot{\boldsymbol{\chi}}_1^\star) \).

\subsection{Multi-Task Control}

\begin{align*}
&\textbf{Single }\text{task: }
\dot{\mathbf{q}} = \mathbf{J}^\dagger \dot{\boldsymbol{\chi}}^\star
\qquad
\textbf{Stacked }\text{use: }
\mathbf{J}_s =
\begin{bmatrix} \mathbf{J}_1 \\ \mathbf{J}_2
\end{bmatrix}\\
&\textbf{Prioritized}\quad
\dot{\mathbf{q}} = \mathbf{J}_1^\dagger
\dot{\boldsymbol{\chi}}_1^\star + (\mathbf{I} - \mathbf{J}_1^\dagger
\mathbf{J}_1) \mathbf{J}_2^\dagger (\dot{\boldsymbol{\chi}}_2^\star -
\mathbf{J}_2 \mathbf{J}_1^\dagger \dot{\boldsymbol{\chi}}_1^\star)
\end{align*}

\subsubsection{Error Analysis and Trajectories}

\textbf{Task error}
$\mathbf{e} = \boldsymbol{\chi}^\star -
\boldsymbol{\chi}$

\textbf{Control}
$\dot{\boldsymbol{\chi}}^\star = \dot{\boldsymbol{\chi}}_d +
\mathbf{K} \mathbf{e}$
(resolved rate)

\textbf{Joint trajectory} Interpolate $\mathbf{q}(t)$\\
(cubic poly: $q(t) = a_0 + a_1 t + a_2 t^2 + a_3 t^3$; match vel/acc)

\ssthl{Exam patterns: Design inverse control loop for redundant arm;
avoid singularities via damping/nullspace.}

